\documentstyle[epsf,11pt]{article}
\begin{document}

\title{ Reducing quasi-ergodicity in a double well potential by Tsallis Monte Carlo simulation }

\author{ Masao Iwamatsu$^{\dagger}$\thanks{Permanent address.  E-mail:iwamatsu@ce.hiroshima-cu.ac.jp} and Yutaka Okabe$^{\dagger}$ \\$^{*}$Department of Computer Engineering, Hiroshima City University\\  Hiroshima 731-3194, Japan\\ and\\ $^{\dagger}$Department of Physics, Tokyo Metropolitan University\\  Hachioji, Tokyo 192-0397, Japan }

\date
{
}

\maketitle

\begin{abstract}
A new Monte Carlo scheme based on the system of Tsallis's generalized
statistical mechanics is applied to a simple double well potential
to calculate the canonical thermal average of potential energy.
Although we observed serious quasi-ergodicity when using the standard
Metropolis Monte Carlo algorithm, this problem is largely reduced by
the use of the new Monte Carlo algorithm. Therefore the ergodicity
is guaranteed even for short Monte Carlo steps if we use this new
canonical Monte Carlo scheme. \\

\begin{flushleft}

PACS: 02.70.Lq; 05.70.-a

Key words: Monte Carlo; Tsallis statistics; double well potential\\

\end{flushleft}
\end{abstract}

\newpage
\section{Introduction}

The ergodic hypothesis is fundamental to statistical mechanics.
This hypothesis states that the time average of an observable
event equals the phase-space average. In practical application,
however, problems can arise in various types of simulations
if the system must overcome high energy barriers to reach other
regions of phase space. In that case, the length of a simulation
needed in order to obtain enough statistical samples of all
regions of phase space may be extremely long. In the Monte Carlo
simulation, the errors arise as a consequence of the finite
length of the Monte Carlo walk. This error can be serious in
canonical Monte Carlo sampling, especially at low temperatures.
This problem, referred as "quasi-ergodicity" by Valleau and
Whittington~\cite{VW}, appears even in the simplest double well
problem, where the two wells are separated by a large
barrier~\cite{FFD}.

Recently several authors~\cite{Pe,TS,AS1} have proposed a
generalized simulated annealing method to locate the global
minimum based on the generalized statistical mechanics
proposed by Tsallis~\cite{Ts}, which can overcome the slow
convergence of the traditional simulated annealing
method~\cite{KGV} based on the standard Monte Carlo scheme.
Subsequently, Andricioaei and Straub have pointed out that
their generalized Monte Carlo (Tsallis Monte Carlo)~\cite{AS2}
method may be used to overcome the quasi-ergodicity appearing
in the canonical average at constant temperatures.

In this short note we use a new Monte Carlo scheme~\cite{AS2,AS3,SA}
based on Tsallis's generalized statistical mechanics~\cite{TS}
for the calculation of the thermal average of the potential energy
of a simple double well potential at constant temperatures.
We examine the performance of this new algorithm designed to
overcome the quasi-ergodicity.

\section{A generalized Monte Carlo scheme}

In the generalized statistical mechanics proposed by
Tsallis~\cite{Ts}, a crucial role is played by the generalized
entropy $S_{q}$ defined as
\begin{equation}
S_{q}=k\frac{1-\sum p_{i}^{q}}{q-1}
\label{eq:1}
\end{equation}
where $q$ is a real number which characterizes the statistical
mechanics, and $p_{i}$ is the probability of states $i$. This
entropy $S_{q}$ becomes the usual Gibbs-Shannon entropy
$S_{1}=-k\sum p_{i}\ln p_{i}$ when $q\rightarrow 1$. By
maximizing the generalized entropy with the constraints
\begin{equation}
\sum p_{i}=1\;\;\;\mbox{and}\;\; \sum p_{i}^{q}\epsilon_{i}=\mbox{const}
\label{eq:2}
\end{equation}
where $\epsilon_{i}$ is the energy spectrum, the statistical weight
for the generalized canonical ensemble characterized by the
parameter $q$ is given by the Tsallis weight
\begin{equation}
p_{i}^{q}=\exp\left(-\overline{\beta}\overline{\epsilon}_{i}\right)/Z_{q}^{q}
\label{eq:3}
\end{equation}
with
\begin{equation}
\overline{\epsilon}_{i}=\frac{q}{\overline{\beta}(q-1)}
\ln\left[1-(1-q)\overline{\beta}\epsilon_{i }\right]
\label{eq:4}
\end{equation}
where $\overline{\epsilon}_{i}$ is the generalized energy and
$\overline{\beta}$ is the Lagrange multiplier, which plays the
role of the generalized temperature in the Tsallis statistical
mechanics. We should note here that the statistical weight is
{\it not} given by the probability distribution $p_{i}$ {\it but}
rather by $p_{i}^{q}$ in order to calculate thermal averages, as
can be seen in (\ref{eq:2}). The generalized canonical partition
function $Z_{q}$ in (\ref{eq:3}) is given by
\begin{equation}
Z_{q}=\sum \exp\left(-\overline{\beta}\overline{\epsilon}_{i}/q\right)
\label{eq:5}
\end{equation}
In the limit $q\rightarrow 1$, $\overline{\epsilon}_{i}\rightarrow
\epsilon_{i}$
and $\overline{\beta}\rightarrow\beta=1/kT$ where $T$ is the usual
temperature~\cite{CT}, then the Tsallis weight (\ref{eq:3}) becomes the
Boltamann weight $p_{i}=\exp(-\beta\epsilon_{i})/Z_{1}$ of the usual
canonical ensemble.

In this generalized statistical mechanics, Andricioaei and
Straub~\cite{AS1} have noted that since the average of an
observable $O$ is defined by
\begin{equation}
<O>_{q}=\sum p_{i}^{q}O_{i},
\label{eq:6}
\end{equation}
the detailed balance condition should be written as
\begin{equation}
p_{i}^{q}W_{ij}=p_{j}^{q}W_{ji}
\label{equation}
\end{equation}
where $W_{ij}$ is the element of the transition matrix. Based
on this observation, they~\cite{AS1} pointed out that the
generalized Monte Carlo algorithm~\cite{TS} originally proposed
by Penna~\cite{Pe} in his generalized simulated annealing method
cannot satisfy the detailed balance. They~\cite{AS1} then
proposed a new generalized Monte Carlo scheme, where the
acceptance probability $p$ of the Monte Carlo move is given by
\begin{equation}
p=\min\left[1,\exp(-\overline{\beta}\Delta \overline{\epsilon})\right]
\label{eq:7}
\end{equation}
which has a similar form to the familiar Metropolis
algorithm~\cite{MRRTT}
\begin{equation}
p=\min\left[1,\exp(-\beta\Delta \epsilon)\right] \label{eq:8}
\end{equation}
where $\Delta\overline{\epsilon}$ is the increase of the
generalized energy (\ref{eq:4}) while $\Delta\epsilon$ is that
of the usual energy. Andricioaei and Straub~\cite{AS1} combined
this Monte Carlo algorithm with the usual simulated annealing
technique~\cite{KGV} to optimize the conformation of tetrapeptides.
They showed that their algorithm is more effective than the
standard simulated annealing based on the usual molecular
dynamics or Monte Carlo methods.

Recently, Andricioaei and Straub~\cite{AS2,AS3} pointed out that their
generalized Monte Carlo scheme~\cite{AS1} for simulated annealing can
also be used to calculate the standard canonical ensemble averages at
constant temperatures
\begin{equation}
<O>_{1}=\frac{\sum O_{i}\exp(-\beta \epsilon_{i})}
{\sum \exp(-\beta \epsilon_{i})}
\label{eq:9}
\end{equation}
using the general rule of the importance sampling~\cite{VW}
\begin{equation}
<f>=\sum f_{i}=\sum \left(\frac{f_{i}}{g_{i}}\right)g_{i}
\label{eq:8b}
\end{equation}
Then, equation (\ref{eq:9}) can be calculated using the generalized
Monte Carlo scheme from
\begin{eqnarray}
<O>_{1}&=&\frac{ \sum O_{i}\exp\left(-\left(\beta\epsilon_{i}-\overline{\beta} \overline{\epsilon}_{i} \right)\right)\exp\left(-\overline{\beta} \overline{\epsilon}_{i}\right) }{\sum \exp\left(-\left(\beta\epsilon_{i}-\overline{\beta} \overline{\epsilon}_{i}\right)  \right)\exp(-\overline{\beta}\overline{\epsilon}_{i})}
\label{eq:10}\\
&=& \frac{< O \exp\left(-\left(\beta\epsilon- \overline{\beta}\overline{\epsilon}  \right)\right)>_{q}}{<\exp\left(-\left(\beta\epsilon- \overline{\beta}\overline{\epsilon}\right)\right)>_{q}}
\label{eq:11}
\end{eqnarray}
where the generalized average $<>_{q}$ is defined by (\ref{eq:6}).
In practice, the equilibrium thermal average in the standard
canonical ensemble can be calculated by conducting these
"generalized" Monte Carlo steps and accumulating the samples by
\begin{equation}
<O>_{1}=\frac{\sum O_{j} \exp\left(-\left(\beta\epsilon_{j}-  \overline{\beta}\overline{\epsilon}_{j}\right)\right)}
{\sum \exp\left(-\left(\beta\epsilon_{j}-
\overline{\beta}\overline{\epsilon}_{j}\right)\right)}
\label{eq:12}
\end{equation}
where the sum $j$ runs over to the generalized Tsallis Monte
Carlo steps produced by eq.(\ref{eq:7}) instead of the usual
Metropolis algorithm (\ref{eq:8}). The equation (\ref{eq:12})
is a generalized expression, and we may choose
$\overline{\beta}=\beta$. Then, the equation (\ref{eq:12})
reduces to
\begin{equation}
<O>_{1}=\frac{\sum O_{j} \exp\left(-\beta\left(\epsilon_{j}-  \overline{\epsilon}_{j}\right)\right)}
{\sum \exp\left(-\beta\left(\epsilon_{j}-
\overline{\epsilon}_{j}\right)\right)},
\label{eq:12x}
\end{equation}
which has been used in \cite{AS2,AS3}. From now on, the actual
calculation will be done by choosing $\overline{\beta}=\beta$.

We would like to point out here that a similar reweighting
technique is used in the histogram method~\cite{FS} and
the multicanonical ensemble method~\cite{BN}. Andricioaei and
Straub applied this Tsallis Monte Carlo algorithm to the
two-dimensional Ising system~\cite{AS2}, a classical one-dimensional
potential and a 13-atom Lennard-Jones cluster~\cite{AS3}, and
showed that this new Monte Carlo algorithm is more effective
than the standard Metropolis algorithm. They also proposed the
generalized molecular dynamics based on Tsallis generalized statistical
mechanics~\cite{AS3,SA}

\section{Application to a classical double well potential}

In order to check the performance of this new Monte Carlo
algorithm~\cite{AS2,AS3} in a classical system such as molecules
and liquids, we look at the problem of a classical particle
in a one-dimensional double well potential defined by
\begin{equation}
V(x)=\delta\left(3 x^{4}+4(\alpha-1)x^{3}-6\alpha x^{2}\right)+1
\label{eq:13}
\end{equation}
where
\begin{equation}
\delta=\frac{1}{2\alpha+1}
\label{eq:14}
\end{equation}
which was examined by Frantz {\it et al.}~\cite{FFD} to demonstrate
the quasi-ergodicity of the standard Metropolis algorithm. The
potential given by (\ref{eq:13}) has the double minimum at $x=1$
and $x=-\alpha$; therefore, it represents a symmetrical double well
when $\alpha=1$ and a single well when $\alpha=0$. This potential
is more conveniently characterized by the parameter $\gamma$~\cite{FFD}
\begin{equation}
\gamma=\frac{V(0)-V(-\alpha)}{V(0)-V(1)}
=\alpha^{3}\left(\frac{\alpha+2}{2\alpha+1}\right)
\label{eq:15}
\end{equation}
In figure 1, we show the asymmetric double well potential $V(x)$
when $\gamma=0.9$ as well as the equilibrium Tsallis weight
$p_{\rm T}\propto \exp(-\beta \overline{V}(x))$, which is renormalized,
and Boltzmann weight $p_{\rm B}=\exp(-\beta V(x))/Z_{1}$. It is obvious
that the statistical weight $p_{\rm T}$ of the Tsallis generalized
statistical mechanics has a greater chance of crossing the barrier
and, therefore, a greater possibility of avoiding quasi-ergodicity.

Alternatively this Tsallis's statistical mechanics can be regarded
as a search-space smoothing method which deforms the rugged potential
landscape by a smoother one. In figure 2, we show the smoothed
potential $\overline{V}(x)$ calculated from (\ref{eq:4}), which has
a smoother landscape and a lower energy barrier than the
original landscape $V(x)$.

We use this double well potential to look at the quasi-ergodicity
of the standard as well as the generalized Monte Carlo scheme by
examining the classical average potential energy~\cite{FFD}
\begin{equation}
<V>=\frac{\int V(x)\exp\left(-\beta V(x)\right)dx}
{\int \exp\left(-\beta V(x)\right)dx }
\label{eq:16}
\end{equation}
which can be calculated rigorously by numerical quadrature as a
function of the temperature $\beta$ and the asymmetry parameter
$\gamma$. We employ the same Monte Carlo procedure of Frantz
{\it et al.}~\cite{FFD} and use the uniform sampling distribution
from previous position $x$ to new position $x^{'}$ given by
\begin{eqnarray}
T(x^{'}|x)&=&\frac{1}{\Delta}\;\;\;\mbox{for}
\;\;|x-x^{'}|<\frac{\Delta}{2}\ \nonumber \\
&=& 0\;\;\;\mbox{otherwise}
\label{eq:17}
\end{eqnarray}
The step size $\Delta$ can be chosen to be large enough for the
one-dimensional problem so that the quasi-ergodicity can
be reduced~\cite{FFD}. However, we follow Frantz {\it et
al.}~\cite{FFD} and use the scaling $\Delta=2.5/\sqrt{\beta}$,
which maintains an approximately 50\% acceptance rate at all
temperatures $\beta$, by considering the application to
multidimensional systems. This 50\% acceptance is commonly
employed in the classical Monte Carlo community~\cite{AT}.

Since the error caused by the quasi-ergodicity depends upon
the walk initialization, we start the Monte Carlo walk at
the global minimum $x=1$ and at the metastable minimum $x=-\alpha$.
If the walk starts at the global minimum, then the average
potential $<V>$ will be low if the walk is quasi-ergodic, as
the distribution associated with the higher energy well around
$\alpha$-minimum will be insufficiently sampled. While the walk
is initialized at $-\alpha$, the walk may be trapped at this
metastable $\alpha$-well and the average potential $<V>$
will be too high.

In figure 3, we show the average potential energy $<V>$ calculated
from (i) the standard Metropolis Monte Carlo and (ii$\sim$iv) the
generalized Tsallis Monte Carlo algorithm as a function of
temperature $\beta$. Similar curves can be found in figure 2 of
reference \cite{FFD}; however, those authors plotted the results
from the usual Metropolis algorithm initialized at the global
minimum $x=1$ and at the random positions. In order to check
the quasi-ergodicity of the walk, we instead show two curves
initialized at the global minimum $x=1$ and the metastable
$\alpha$-minimum at $x=-\alpha$ for each scheme. The average
potential $<V>$ is obtained from 100 independently initialized
walks, each consisting of 500 warm-up steps followed by $10^{4}$
steps with data accumulation. The average energy $<V>$ calculated
from the standard Metropolis algorithm shows large errors due to
the quasi-ergodicity. In particular, the walk initialized at
the metastable $\alpha$-minimum greatly overestimates the potential
energy. The walk starting from the global minimum also shows
noticeable error and underestimates the average energy, as expected.

In contrast to the results from the standard Metropolis algorithm,
the results from the generalized Monte Carlo algorithm~\cite{AS2,AS3}
show excellent agreement with the numerically exact result. The
walks started from the global minimum $x=1$ produce the average
energy which is almost indistinguishable from the exact result up
to the lowest temperature (high $\beta$) examined. Even the walk
started from the metastable $\alpha$ minimum produces the energies
that are fairly accurate up to rather low temperature $\beta \sim 15$
for $q=1.5$, which shows that this generalized algorithm does
a much better job than the standard Monte Carlo scheme. The agreement
with the exact result is further improved if we increase the value
of parameter $q$ from $1.5$ (see Figure 3). It should be noted
that the probability distribution associated with the parameter
$q=2$ is given by the Cauchy distribution and with $q>5/3$ by the
super-diffusive L\'{e}vy distribution~\cite{TS}. The theoretical upper
boundary is $q=3$. The quasi-ergodicity can be largely circumvented
by the use of Tsallis's generalized statistical mechanics with
$q>1$, as shown in figure 3.

In order to overcome the quasi-ergodicity, the Monte Carlo
walk has to pass many times through the energy barrier.
Figure 4 compares the time series of the Monte Carlo walk
generated by the standard Monte Carlo scheme ($q=1$) and that
of the generalized Monte Carlo scheme. Similar diagrams
can be found in reference ~\cite{AS3}. Although the
trajectory of the standard Monte Carlo scheme is mostly
confined within the metastable $\alpha$ minimum, the number
of barrier crossing of the Tsallis Monte Carlo scheme is
significantly greater. This barrier crossing occurs more
often when the parameter $q$ is larger. The quasi-ergodicity
can be removed by Tsallis's generalized Monte Carlo scheme
by increasing the magnitude of the parameter $q$ from $1$.

In figure 5, we show the average energy $<V>$ as a function of
the asymmetry parameter $\gamma$. Again a similar diagram can
be found in reference~\cite{FFD}. Here again we can
clearly observe a large error for the initialization at the
metastable $\alpha$ minimum when the standard Monte Carlo
procedure is used. In this case, the superiority of Tsallis's
generalized Monte Carlo scheme over the traditional one is
also obvious.

In table 1 we showed the effect of quasi-ergodicity in the
convergence when $\gamma=0.9$ and $\beta=10$ for both
the standard Monte Carlo and the generalized Monte Carlo
algorithms with $q=1.5$, each originating at the metastable
$\alpha$-minimum. We show the average energy $<V>$ and the
standard deviation (STD) obtained from 100 independently
initialized walks. We found that the problem of quasi-ergodicity
obtained with the standard Metropolis algorithm is still present,
even if we increase the number of Monte Carol steps. On the other
hand the result from the generalized Monte Carlo simulation is
significantly improved by increasing the number of Monte Carlo
steps, and ergodicity can be recovered. Furthermore, STD
decreases as $\sim 1/\sqrt{N}$ when the Monte Carlo step $N$ is
increased for the generalized Monte Carlo simulation, while such
a systemic decrease is not observed for the standard Monte
Carlo simulation.

This convergence can be clearly visualized by plotting the
visiting probability $P(x)$ of the Monte Carlo walks, which
should be given by the Boltzmann weight for the standard
Monte Carlo scheme and by the Tsallis weight (\ref{eq:3}) for
the generalized Monte Carlo scheme. In figure 6, we compare
$P(x)$ generated by the standard and generalized Monte Carlo
walks with the Boltzmann and the Tsallis weights. The visiting
probability $P(x)$ converges rapidly to the Tsallis weight as
the number of the Mone Carlo walks is increased for the
generalized Monte Carlo scheme. On the other hand, $P(x)$
recovers the Boltzmann weight only when the Monte Carlo steps
are increased upto $\sim 10^{6}$ in the standard Monte
Carlo scheme.

It is also interesting to discuss the convergence in the context of
the "generalized ergodic measure" of Thirumalai, Mountain, and
Kirkpatrick~\cite{TMK,AS3}. Using the energy metric, the ergodic measure
$d_{V}(n)$ is defined by
\begin{equation}
d_{V}(n)=\left(V^{a}(n)-V^{b}(n)\right)^{2},
\label{eq:18}
\end{equation}
where $V^{a}(n)$ and $V^{b}(n)$ are the move average of the potential
energy $V$ along the trajectories $a$ and $b$, which
are defined from (\ref{eq:12x}) by
\begin{eqnarray}
V^{a}(n)&=&\frac{\sum_{j=0}^{n} V_{j}^{a} \exp\left(-\beta\left(V_{j}^{a}-  \overline{V}_{j}^{a}\right)\right)}
{\sum_{j=0}^{n} \exp\left(-\beta\left(V_{j}^{a}-
\overline{V}_{j}^{a}\right)\right)}.
\label{eq:19}
\end{eqnarray}
for the generalized Tsallis Monte Carlo scheme, and
\begin{eqnarray}
V^{a}(n)&=&\frac{\sum_{j=0}^{n} V_{j}^{a} \exp\left(-\beta V_{j}^{a}\right)}
{\sum_{j=0}^{n}1}.
\label{eq:19x}
\end{eqnarray}
for the standard Monte Carlo scheme.
For an ergodic system, Thirumalai {\it et al}.~\cite{TMK} suggested that
the ergodic measure converges as $d_{V}(n)\rightarrow \infty$ if
$n\rightarrow \infty$. They found that
\begin{equation}
d_{V}(n)\simeq d_{V}(0)\frac{1}{D_{V}n}
\label{eq:20}
\end{equation}
where the "diffusion" constant $D_{V}$ depends on temperature. In figure
7, we show $d_{V}(0)/d_{V}(n)$ for $\gamma=0.9$ potential obtained from 100
independent pairs ($a$, $b$) of Monte Carlo walks as a function of Monte Carlo
steps $n$. We chose the trajectory starting from the metastable minimum
$x=-\alpha$ (=-0.9163) and that from the stable minimum $x=1$,
respectively, as the two independent trajectories $a$ and $b$. It is
clear from figure 7 that the normalized inverse of the ergodic measure
$d_{V}(0)/d_{V}(n)$ grows linearly with the Monte Carlo steps $n$.
Therefore, the ergodic measure $d_{V}(n)$ decreases according to
(\ref{eq:20}), and the diffusion constant $D_{V}$ can be determined.

Figure 8 shows the diffusion constant $D_{V}$ as a function of temperature
$\beta$. The diffusion constant is a decreasing function of the inverse
temperature $\beta$; it is more difficult to recover the ergodicity when
the temperature is lowered. However, at lower temperatures $\beta>20$ with
$\beta\Delta V>1$ where $\Delta V=V(-\alpha)-V(1)=0.1$, the diffusion
constant starts to increase again for the Tsallis Monte Carlo walks ($q\neq
1$). This is due to the fact that the equilibrium thermal distribution
around the metastable minimum at $x=-\alpha$ becomes negligibly small at
such low temperatures, and the ergodicity of the trajectory among the
stable and the metastable minima is
irrelevant once trajectories $a$ and $b$ both fall into the stable
minimum around $x=1$.

It is apparent from figure 8 that the diffusion constant $D_{V}$ obtained
from the standard Monte Carlo algorithm obeys the usual activation form
\begin{equation}
D_{V} \sim \exp\left(-\beta E_{b}\right)
\label{eq:21}
\end{equation}
where the activation energy $E_{b}$ is estimated to be $E_{b}\simeq 0.4$,
which is the same order of magnitude as the barrier height ($\sim 1$)
between two wells, as shown in figure 1. On the other hand, the
diffusion
constant $D_{V}$ obtained from the Tsallis generalized Monte Carlo scheme is
characterized by a power-law form
\begin{equation}
D_{V} \sim \beta^{-\eta}
\label{eq:22}
\end{equation}
with the exponent $\eta$. This result is conceivable, based on the
acceptance
probability (\ref{eq:7}) using the Tsallis weight, because
\begin{eqnarray}
D_{V} &\sim& \exp(-\beta\bar{E}_{b})
\nonumber \\
&\sim& \left[1-(1-q)\beta E_{b} \right]^{-q/(q-1)}
\label{eq:23}
\end{eqnarray}
and $D_{V}\sim \beta^{-q/(q-1)}$. We note in figure 8 that $\eta\sim 1.6$
for $q=1.5$ and $\eta\sim 0.85$
for $q=2.5$. These values are close to $q/(q-1)=3$ for $q=1.5$ and
$q/(q-1)=1.66$ if $q=2.5$. Therefore, the diffusion constant decreases
more slowly as a function of the inverse temperature $\beta$ as
$\beta^{-\eta}$ for the
generalized Monte Carlo scheme, though it decreases exponentially
faster for
the standard Monte Carlo scheme. The exponent $\eta$ becomes smaller as
$q$ is increased and the diffusion is possible even at low temperatures.
This result can be anticipated based on data presented in figures 3
and 6. This power-law temperature dependence of the diffusion constant
$D_{V}$ for $q\neq 1$ has been pointed out by Straub and
Andricioaei~\cite{SA2}.

\section{Concluding remarks}

In conclusion, we found that the so-called "quasi-ergodicity"
in a double well potential encountered in the standard Metropolis
Monte Carlo algorithm is largely circumvented by the
use of the generalized Monte Carlo algorithm of Andricioaei
and Straub~\cite{AS2,AS3} based on Tsallis's~\cite{TS} generalized
statistical mechanics. Therefore this algorithm will be useful
in the Monte Carlo study of any system, in particular, when
the first order phase transition occurs. We believe that
this algorithm will be useful, for example, in the study of
the melting transition of clusters for which the J-walk
algorithm~\cite{FFD} and the multiple histrogram
method~\cite{WA} have been utilized. It is also interesting
to apply this new algorithm to the problems where the slow
dynamics due to randomness or frustration is serious, for
example, the spin-glass problems. This generalized
Monte Carlo algorithm based on Tsallis's statistical mechanics
use the reweighting technique similar to the multicanonical
method~\cite{BN}. But this algorithm is much simpler than
the multicanonical methods and appears to be more promising
because it does not require an iterative determination of the
weighting factor by preliminary Monte Carlo runs.

Recently, a new formulation of the Tsallis statistics
has been proposed~\cite{TMP}. With a new choice of energy
constraint, unfamiliar consequences in the previous formulation
have been erased. However, the Lagrange parameter used
in the new formulation can be related to that in the previous
formulation. Therefore, in performing Monte Carlo simulations
with reweighting technique, we may use the previous formulation
of the Tsallis statistics.

\begin{flushleft}
{\Large \bf Acknowlegement}
\end{flushleft}

Authors are grateful to the reviewer for several useful comments
which were incorporated into the final version of the manuscript.
M. I. acknowledges support from the Hiroshima City University
Grant for Special Academic Research. He is also grateful
to the Department of Physics, Tokyo Metropolitan University,
where this work was initiated, for its hospitality to him.

\begin{thebibliography}{16}

\bibitem{VW} J. P. Valleau and S. G. Whittington, in:Statistical Mechanics,
edited by B. J. Berne (Plenum Press, New York, 1977), Ch. 4.
\bibitem{FFD} D. D. Frantz, D. L. Freeman and J. D. Doll, J. Chem. Phys.
{\bf 93} (1990)
2769.
\bibitem{Pe} T. J. P. Penna, Phys. Rev. E {\bf 51} (1995) R1.
\bibitem{TS} C. Tsallis and D. A. Stariolo, Physica A {\bf 233} (1996) 395.
\bibitem{AS1} I. Andricioaei and J. E. Straub, Phys. Rev. E {\bf 53}
(1996) R3055.
\bibitem{Ts} C. Tsallis, J. Stat. Phys. {\bf 52} (1988) 479.
\bibitem{KGV} S. Kirkpatrick, C. D. Gelatt, and M. P. Vecchi, Science {\bf
220} (1983) 671.
\bibitem{AS2} I. Andricioaei and J. E. Straub, Physica A {\bf 247} (1997) 553.
\bibitem{AS3} I. Andricioaei and J. E. Straub, J. Chem. Phys. {\bf 107}
(1997) 9117.
\bibitem{SA} J. E. Straub and I. Andricioaei, Braz. J. Phys. {\bf 29}
(1999) 179.
\bibitem{CT} E. M. F. Curado and C. Tsallis, J. Phys. A {\bf 24} (1991) L69.
\bibitem{MRRTT} N. Metropolis, A. W. Rosenbluth, N. N. Rosenbluth, A. H. Teller 
and E. Teller, J. Chem. Phys. {\bf 21} (1953) 1087.
\bibitem{FS} A. M. Ferrenberg and R. H. Swendsen, Phys. Rev. Lett. {\bf
61} (1988) 2635; R. H.
Swendsen, J. S. Wang, and A. M. Ferrenberg, The Monte Carlo Methods in
Condensed Matter
Physics, edited by K. Binder (Springer Verlag, Berlin, 1992), pp. 75-91.
\bibitem{BN} B. Berg and T. Neuhaus, Phys. Lett. B {\bf 267} (1991) 249.
\bibitem{AT} M. P. Allen and D. J. Tildesley, Computer Simulation of Liquids
(Clarendon Press, Oxford, 1987), pp.121-123.
\bibitem{TMK} D. Thirumalai, R. D. Mountain and T. R. Kirkpatrick, Phys.
Rev. A {\bf 39} (1989) 3563.
\bibitem{SA2} J. E. Straub and I. Andricioaei, "Exploiting Tsallis
statistics,'' in: Algorithms for Macromolecular Modeling, Edited by P.
Deuflhard, J. Hermans, B. Leimkuhler, A. Mark, S. Reich and R. D. Skeel,
Lecture Notes in Computational Science and Engineering (Springer Verlag
Berlin, 1998), vol. 4, pp. 189-204.
\bibitem{WA} S. Weerasinghe and F. G. Amar, J. Chem. Phys. {\bf 98} (1993)
4967.
\bibitem{TMP} C. Tsallis, R.S. Mendes, and A.R. Plastino, Physica A {\bf
261} (1998) 534.
\end{thebibliography}

\newpage

\begin{figure}[th]
\epsfxsize=8cm
 \centerline{\epsfbox{tfig1.eps}}
\caption{A classical double well  potential (16) (thin solid curve),for $\gamma=0.9$, the Tsallis weight  $p_{\rm T} \propto \exp(-\beta \overline{V}(x))$ with $q=1.5$  (dotted curve), $q=2.5$ (chain curve) and the Boltzmann  weight $p_{\rm B}=\exp(-\beta V(x))/Z_{1}$ (thick solid curve)  both at temperature $\beta=10$. The Tsallis weights are  renormalized.}
\label{fig:1}
\end{figure}

\begin{figure}[th]
\epsfxsize=8cm
 \centerline{\epsfbox{tfig2.eps}}
\caption{A smoothed potential landscape $\overline{V}(x)$  when $q=1.5$ (dotted curve) and $q=2.5$ (chain curve) for  $\gamma=0.9$ and $\beta=10$, compared with the original  potential landscape $V(x)$ ($q=1$, thin solid curve).  The energy barrier is lowered by the smoothing defined  by (3) of Tsallis's statistical mechanics.}
\label{fig:2}
\end{figure}

\begin{figure}[th]
\epsfxsize=8cm
 \centerline{\epsfbox{tfig3.eps}}
\caption{The classical average potential energy $<V>$ for  the $\gamma=0.9$ potential as a function of $\beta$. The  thin solid curves are initialized at the  metastable $\alpha$-minimum $x_{0}=-0.9613$  with the standard Monte Carlo  algorithm with the parameter $q=1$ and the generalized Monte  Carlo algorithm with $q=1.5$, $q=2.0$ and $q=2.5$.  The exact result is represented by the thick solid  curve. The results initialized at the global minimum $x_{0}=1$  and calculated from the Metropolis algorithm with $q=1$ and that from  the generalized Monte Carlo algorithm with $q=1.5$ are shown  by the dotted curve. All results are obtained from 100 independently  initialized Monte Carlo walks, each consisting of 500 warm-up steps  followed by $10^{4}$ steps with data accumulation. Some  representative error bars are  also shown.}
\label{fig:3}
\end{figure}

\begin{figure}[th]
\epsfxsize=8cm
 \centerline{\epsfbox{tfig4a.eps}}
\vspace{4mm}
\epsfxsize=8cm
 \centerline{\epsfbox{tfig4b.eps}}
\vspace{4mm}
\epsfxsize=8cm
 \centerline{\epsfbox{tfig4c.eps}}
\caption{ Time series of the Monte Carlo walks generated by  (a) the standard Monte Calro ($q=1$), (b) the Tsallis Monte  Carlo with $q=1.5$, and (c) $q=2.5$ when $\beta=10$ and  $\gamma=0.9$. All walks start from the metastable  $\alpha$-minimum, and 500 warm up steps are also included.  The number of barrier crossing is greater for larger $q$.}
\label{fig:4}
\end{figure}

\begin{figure}[th]
\epsfxsize=8cm
 \centerline{\epsfbox{tfig5.eps}}
\caption{The average potential energy $<V>$ for the temperature  $\beta=10$ as a function of the asymmetry parameter $\gamma$ of  the double well potential. The solid and dotted curves are used the same as in figure 3. We only show the result and some  representative error bars for the cases $q=1$ and $q=1.5$ of figure 3.}
\label{fig:5}
\end{figure}

\begin{figure}[th] 
\def\halftext{.471\textwidth} 
\parbox{.471\textwidth}{  \epsfxsize=.471\textwidth   \epsfbox{tfig6a.eps}  } 
\hspace{4mm} 
\parbox{.471\textwidth}{  \epsfxsize=.471\textwidth   \epsfbox{tfig6b.eps}  }
\caption{The Visiting probabilities $P(x)$ obtained from (a) the standard Monte Carlo walk compared with the Boltzmann weight, and (b) the generalized Monte Carlo  walks compared with the Tsallis weight (c.f. figure 1). Probability distribution $P(x)$ generated from the generalized Monte Carlo scheme can recovers the  equilibrium Tsallis weight within short Monte Carlo  steps (a), while it cannot recover the equilibrium  Boltzmann weight until rather long ($\sim 10^{6}$) Monte  Carlo steps are executed (b).}
\vspace*{20mm}
\label{fig:6}
\end{figure}

\begin{figure}[th]
\epsfxsize=8cm
 \centerline{\epsfbox{tfig7.eps}}
\caption{The normalized inverse of the generalized ergodic  measure $d_{V}(0)/d_{V}(n)$ as a function of the number of Monte Carlo steps $n$ calculated for $\gamma=0.9$  potential at $\beta=10$ using the standard Monte Carlo scheme ($q=1$) and the generalized Monte Carlo scheme ($q=1.5, 2.5$). $d_{V}(0)/d_{V}(n)$ grows linearly with the Monte Carlo steps $n$ and the diffusion is faster  for larger $q$.}
\label{fig:7}
\end{figure}

\begin{figure}[th] 
\def\halftext{.471\textwidth} 
\parbox{.471\textwidth}{  \epsfxsize=.471\textwidth   \epsfbox{tfig8a.eps}  } 
\hspace{4mm} 
\parbox{.471\textwidth}{  \epsfxsize=.471\textwidth   \epsfbox{tfig8b.eps}  }
\caption{The diffusion constant $D_{V}$ as a function of  temperature $\beta$. (a) A straight line of the semi-log plot  of $D_{V}$ vs $\beta$ for the standard Monte Carlo scheme  indicates activation-type diffusion, while (b) straight  lines of the log-log plot for the Tsallis Monte Carlo scheme  in the high temperature region indicate slow power-law  decrease of the diffusion constant as the temperature is lowered (larger $\beta$).}
\vspace*{20mm}
\label{fig:8}
\end{figure}

\newpage
\begin{table}[th]
\caption{The convergence of the Metropolis and the generalized  Tsallis Monte Carlo algorithms at $\beta=10$. The average energy  $<V>$, the standard deviation (STD) obtained from 100 independently initialized walks, each originating at the  metastable $\alpha$-minimum.}

\vspace{4mm}
\begin{tabular}{c|c|c}
\hline
&Metropolis Monte Carlo & Tsallis Monte Carlo ($q=1.5)$ \\
steps & $<V>$ (STD) & $<V>$ (STD) \\
\hline
$10^{2}$ & $0.1500$ ($0.0177$) &$0.1065$ ($0.0518$) \\
$10^{3}$ & $0.1520$ ($0.0049$) &$0.0937$ ($0.0394$) \\
$10^{4}$ & $0.1475$ ($0.0192$) &$0.0798$ ($0.0162$) \\
$10^{5}$ & $0.1052$ ($0.0320$) &$0.0798$ ($0.0049$) \\
$10^{6}$ & $0.0805$ ($0.0124$) &$0.0800$ ($0.0017$) \\
Exact & $0.0799$ & $0.0799$ \\
\hline
\end{tabular}
\end{table}

\end{document}