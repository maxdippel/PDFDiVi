\documentstyle[prb,aps,epsf,rotate,amstex,amssymb]{revtex}
\begin{document}
\draft
\makeatletter
\title{Field distributions and effective-medium approximation  for weakly nonlinear media}
\author{Yves-Patrick Pellegrini\thanks{e-mail: {\protect\tt   pellegri@bruyeres.cea.fr}}}
\address{ Service de Physique de la Mati\`{e}re Condens\'{e}e, \\  Commissariat \`{a} l'Energie Atomique,\\  BP12, 91680 Bruy\`{e}res-le-Ch\^{a}tel, France.}
\date{Last modified: November 18, 1999. Printed: \today}
\maketitle
\makeatother
\begin{abstract}
An effective-medium theory is proposed for random weakly nonlinear dielectric
media. It is based on a new gaussian approximation for the probability
distributions of the electric field in each component of a multi-phase
composite. These distributions are computed to linear order from a
Bruggeman-like self-consistent formula. The resulting effective-medium formula
for the nonlinear medium reduces to Bruggeman's in the linear case. It is exact
up to second order in a weak-disorder expansion, and close to the exact result
in the dilute limit (in particular, it is exact for $d=1$ and $d=\infty$). In a
high contrast situation, the noise exponents are $\kappa=\kappa'=0$ near the
percolation threshold. Numerical results are provided for different
weak nonlinearities. The use of the Bruggeman formula as a starting point
for nonlinear homogeneization theories in dimensions $d>2$ is questioned on
the basis of known exact bounds on the noise exponents.
\end{abstract}
\bigskip

\pacs{PACS numbers: 77.84.Lf, 77.84.-s, 77.20.Ht, 05.40.-a}
\makeatother
\section{Introduction}
\label{intro}
Numerous papers have been devoted to the problem of determining the effective
constitutive law of random nonlinear composites in electrostatics
\cite{AHAR87,STRO88,ZENG88,BERG89,BLUM89,KOTH90,PONT92,GU92,YU93,YU94,LEE95,GAO96,WAN96,SALI97,PONT97,BART98},
of importance for understanding
various nonlinear phenomena, either entering the class
of ``weakly nonlinear'' (WNL) phenomena, or that of ``strongly nonlinear''
phenomena such as random fuse-type or dielectric breakdown-type failure of
materials \cite{DUXB86}.

A convenient approach to the homogeneization problem is the ``energetic''
one \cite{HILL63,PONT92b}, which can be summarized as follows. Let $w_x(E)$
be the local potential (the energy density) from which derives the local
nonlinear constitutive law. The homogeneization step consists in computing
the effective potential $W$ as the volume average
\begin{equation}
W=\langle w_x(E)\rangle,
\end{equation}
from which the effective constitutive law can be deduced (see below).
Denoting by $p_\alpha$ the volume proportion of component (``phase'')
$\alpha$ in a composite medium where $w_x=w_\alpha$ if $x$ is in phase
$\alpha$, and introducing the volume averages {\em per phase}
$\langle\cdot\rangle_\alpha$, $W$ can be written
\begin{equation}
\label{weff}
W=\sum_\alpha p_\alpha \langle w_\alpha(E)\rangle_\alpha.
\end{equation}

A class of effective-medium theories, mostly concerned with weak
power-law nonlinearities, use our ability to compute the second moment
of the electric field $E$ in the various phases of the composite
\cite{BERG78}, completed by a decoupling approximation \cite{ZENG88,HUI95a}
\begin{equation}
\label{decoupl}
\langle w_\alpha(E)\rangle_\alpha\simeq
w_\alpha\bigr(\langle E^2\rangle_\alpha^{1/2}\bigl).
\end{equation}
This approximation is exact only when $w_\alpha$ is quadratic
(the linear theory), or when the field is uniform, which occurs only
if all the $w_\alpha$ are equal, i.e.\ in a homogeneous medium.

The purpose of this paper is to present a means to overcome this
approximation, starting from the following remark: actually, with
an additional hypothesis of ergodicity {\em per phase}, a volume phase
average can be reinterpreted as an average over some probability
distribution for the electric field in the phase under consideration.
In this paper, volume averages will henceforth be identified to
statistical averages. Then, the problem of computing $W$,
Equ.\ (\ref{weff}), reduces to computing the probability distribution
${\cal{P}}_\alpha$ of the field in each phase, by means of which the
phase averages $\langle\cdot\rangle_\alpha$ can be carried out,
without appealing to (\ref{decoupl}).

Note that the full probability distribution ${\cal{P}}({\mathbf E})$
of the electric field in the medium is then
\begin{equation}
\label{multimod}
{\cal{P}}({\mathbf E})=\sum_\alpha p_\alpha{\cal{P}}_\alpha({\mathbf E}).
\end{equation}
Thus, the probability distributions of the field {\em per phase} are
functions of great importance, since they carry all the information
needed to build up the effective nonlinear properties. A few analyses
of ${\cal{P}}({\mathbf E})$ are available \cite{BART97}, mainly focused
on its tail of relevance for breakdown phenomena \cite{DUXB95}. However,
up to my knowledge, none has been specifically aimed at the individual
components ${\cal{P}}_\alpha({\mathbf E})$. Chen and Sheng studied
numerically the distribution of the {\em local} field in binary
composites \cite{CHEN91}. An analytical model of the same distribution
was subsequently proposed by Barth\'{e}l\'{e}my and Orland, with excellent
agreement with the latter simulations \cite{BART97}. Both works were
limited to linear composites, in a weak-contrast situation. A first
numerical attempt in the continuum, in a strong-contrast situation
this time, is due to Cheng and Torquato \cite{CHEN97}. They carried
out two-dimensional numerical calculations on disordered systems with
inclusions of various shapes. An exact calculation for ${\cal{P}}$ in
a {\em linear} medium with Hashin-Shtrikman (HS) geometry was recently
made using a density of states approach \cite{CULE98}. In these
works, the distribution ${\cal{P}}$ was found to be essentially
bimodal for a binary medium (save in the HS geometry where
fine-structure peaks are not wiped out by positional disorder),
especially at high dielectric contrast \cite{CHEN97}. This is
understandable in wiew of the decomposition (\ref{multimod}).

Still, in parallel with the second moment of the field, the
{\em first} moment in each phase (the average field) can be computed
as well from any linear effective-medium theory, without requiring
any {\it a priori} knowledge about ${\cal{P}}_\alpha$. Widely used in
mechanics of random continuous media in the framework of the
so-called ``thermoelastic problem'' \cite{WILL81}, this fact seems
to have escaped most of the litterature on random dielectric media,
excepted in the work by Ponte Casta\~{n}eda and Kailasam \cite{PONT97}
where it is used to compute reference fields around which non-linear 
potentials are expanded prior to homogeneization. Knowing the first
two moments is enough to attempt a modelization of ${\cal{P}}_\alpha$.
The proposal examined in the present paper is to approximate each
${\cal{P}}_\alpha({\mathbf E})$ by a gaussian {\em vector}
distribution, and thereby, to estimate (\ref{weff}) for any potential,
without appealing to the strong (and somewhat artificial) decoupling
assumption (\ref{decoupl}).

The scope of the paper is limited however to WNL media, for which it
is sufficient to compute the field distribution to linear order only
\cite{AHAR87,STRO88}. An extension of these considerations to
strongly nonlinear media will be examined elsewhere.

\section{General framework}
\label{th}
Let us therefore consider a $d$-dimensional random medium described
by WNL thermodynamic potentials of the type
\begin{equation}
\label{orig}
w_x(E)=w_x^{\text{lin}}(E) +w_x^{\text{nl}}(E),
\end{equation}
where
\begin{equation}
w_x^{\text{lin}}(E)=\frac{1}{2} \varepsilon E^2
\end{equation}
($\varepsilon(x)$ being the local permittivity entering the linear part),
and $w_x^{\text{nl}}(E)$ is the nonlinear (i.e.\ non-quadratic) part of
the potential, assumed to be small. The local constitutive law is
\begin{equation}
D_i=\frac{\partial w_x(E)}{\partial E_i}.
\end{equation}

The potential is prescribed in each phase: $w_x(E)=w_\alpha(E)$ if $x\in\alpha$.
Under the ergodic hypothesis introduced above, and a ``boundary'' condition
\cite{NOTE1} $\langle{\mathbf{E}}\rangle={\mathbf{E}}^0$, the effective
potential reads
\begin{equation}
W(E^0)=W_{\text{lin}}(E^0)+W_{\text{nl}}(E^0),
\end{equation}
where
\begin{eqnarray}
\label{lin}
W_{\text{lin}}(E^0)&=&\frac{1}{2}\sum_\alpha p_\alpha \varepsilon_\alpha
\langle E^2\rangle_\alpha,\\
\label{nonlin}
W_{\text{nl}}(E^0)&=&\sum_\alpha p_\alpha \langle w_\alpha^{\text{nl}}(E)
\rangle_\alpha.
\end{eqnarray}
The effective constitutive law in the homgeneized medium is then
\begin{equation}
\label{constithom}
D^0_i={\partial W(E^0)\over \partial E^0_i}.
\end{equation}

To first order in $w_x^{\text{nl}}$, the field distributions entering
(\ref{lin}) and (\ref{nonlin}) need only be computed at the linear
level \cite{AHAR87,STRO88}. The procedure is the following one. An
extended linear {\em anisotropic} homogeneization problem is first
considered, with an artificial ferroelectric-like part added to the
potential \cite{PONT97}:
\begin{equation}
\label{ferroel}
\tilde{w}^{\text{lin}}_x(E)=\frac{1}{2} E_i\varepsilon_{ij} E_j+
P_{i} E_i,
\end{equation}
where $\varepsilon_{ij}(x)=\varepsilon^\alpha_{ij}$ and $P_i(x)=
P_i^\alpha$ if $x\in\alpha$. The corresponding local constitutive
law is $D_i=\varepsilon_{ij} E_j+P_i$. Assuming one knows an estimate
for
\begin{equation}
\tilde{W}_{\text{lin}}(E^0)=\langle \tilde{w}^{\text{lin}}_x(E)\rangle,
\end{equation}
(cf.\ below), the following {\em exact} relations hold for the distribution
${\cal{P}}_\alpha$ of the original linear problem (\ref{lin}):
\begin{mathletters}
\label{avfield}
\begin{eqnarray}
\label{ave}
\left\langle E_i\right\rangle_{\alpha}&=&\frac{1}{p^\alpha}\left.
\frac{\partial \tilde{W}_{\text{lin}}}{\partial P^\alpha_i}
\right|_{\varepsilon_{ij}=\varepsilon \delta_{ij}\atop P_i=0},\\
\label{avee}
\left\langle E_i E_j\right\rangle_{\alpha}&=&\frac{2}{p^\alpha}
\left.\frac{\partial \tilde{W}_{\text{lin}}}{\partial \varepsilon^\alpha_{ij}}\right|_{\varepsilon_{ij}=\varepsilon  \delta_{ij}\atop P_i=0}.
\end{eqnarray}
\end{mathletters}
I emphasize that the derivatives are computed at the point $P_i=0$,
$\varepsilon^\alpha_{ij}=\varepsilon^\alpha \delta_{ij}$ where
$\varepsilon^\alpha$ is the scalar permittivity in (\ref{orig})
and (\ref{lin}). There, $\tilde{w}^{\text{lin}}_x=w^{\text{lin}}_x$
and $\tilde{W}_{\text{lin}}=W_{\text{lin}}$. Note that (\ref{avee})
is a slight generalization of the formula \cite{BERG78}
\begin{equation}
\label{derivscal}
\langle E^2\rangle_\alpha=\frac{2}{p^\alpha}\frac{\partial  W_{\text{lin}}}{\partial\varepsilon^\alpha}.
\end{equation}
Both (\ref{ave}) and (\ref{avee}) are particular cases of a more
general theorem, the demonstration of which can be found in appendix
B of the review by Ponte Casta\~{n}eda and Suquet on nonlinear
composites\cite{PONT98}. This theorem is a consequence of the
equality $\langle D_i E_i\rangle=\langle D_i\rangle \langle E_i\rangle$
which holds only for particular boundary conditions, or in the
infinite-volume limit if there exists an effective permittivity.

Let us set
\begin{eqnarray}
M_i^\alpha&=&\left\langle E_i\right\rangle_{\alpha},\\
C^\alpha_{ij}&=&\left\langle E_i E_j\right\rangle_{\alpha}-
M_i^\alpha M_j^\alpha,
\end{eqnarray}
with $M^\alpha_i\equiv M^\alpha \hat{E}^0_i$ and $C^\alpha_{ij}
\equiv C^\alpha_\parallel \hat{E}^0_i\hat{E}^0_j+
C^\alpha_\perp(\delta_{ij}-\hat{E}^0_i\hat{E}^0_j)$, since the
averages only depend on the direction $\hat{E}^0_i=E^0_i/E^0$
of the macroscopic field. I introduce now the gaussian approximation
for the probability distributions ${\cal{P}}_\alpha$
\begin{equation}
\label{pae}
{\cal{P}}_\alpha({\bf E})={1\over \left[(2\pi)^d \det(C^\alpha) \right]^{1/2}} e^{-{1\over 2}(E_i-M^\alpha_i) {C^\alpha}^{-1}_{ij}(E_j-M^\alpha_j)},
\end{equation}
which constitutes the main ingredient of the theory.

There remains to compute an approximation for the quantities
$\left\langle E_i\right\rangle_{\alpha}$ and $\left\langle
E_i E_j\right\rangle_{\alpha}$, in oprder to completely define
the above gaussian distributions. Let us therefore estimate
$\tilde{W}_{\text{lin}}(E^0)$ through an extension of the
Bruggeman self-consistent effective-medium approximation\cite{BRUG35}
to potentials of the form (\ref{ferroel}). The formula is
given and discussed in Appendix A. Using the scalar
permittivities and $P_i=0$, this estimate reduces to
\begin{equation}
W_{\text{lin}}(E^0)=\frac{1}{2}\varepsilon_0 {E^0}^2,
\end{equation}
where $\varepsilon_0$ is the usual Bruggeman effective
permittivity\cite{BRUG35}. In addition, the derivatives
(\ref{ave}) and (\ref{avee}) of the estimate for
$\tilde{W}_{\text{lin}}(E^0)$ yield
\begin{eqnarray}
M^\alpha&=&\mu^\alpha E^0,\\
\label{ee}
\left\langle E_i E_j\right\rangle_\alpha&=&z
{\mu^\alpha}^2\Biggl[E^0_iE^0_j -\frac{{E^0}^2}{d+2}\left(1-y\langle\mu^2 \rangle\right)\delta_{ij}\Biggr],
\end{eqnarray}
where
\begin{eqnarray}
\mu^\alpha&=&\frac{d\varepsilon^0}
{\varepsilon^\alpha+(d-1)\varepsilon^0},\\
y&=&\frac{\varepsilon_0}{\langle\mu^2\varepsilon\rangle},\\
z&=&\frac{d+2}{d\langle\mu^2\rangle+2/y}.
\end{eqnarray}
Note that the Bruggeman equation for $\varepsilon_0$ reads
$\langle\mu\rangle=1$, or alternatively $\langle\mu
\varepsilon\rangle=\langle\mu\rangle\varepsilon_0=\varepsilon_0$.
From these relations, we check that: (i) $\langle\varepsilon
E\rangle=\varepsilon^0 \langle E\rangle$; (ii) $\langle\varepsilon
E^2\rangle=\varepsilon^0 \langle E\rangle^2$, where $\langle
E\rangle=E^0$, as must be. Expressions for $C^\alpha_\parallel$
and $C^\alpha_\perp$ are readily obtained from (\ref{ee}).

In Fig.\ 1 are displayed curves for the probability density
${\cal{P}}_\parallel(E_\parallel/E^0)$
of the normalized longitudinal electric field,
$E_\parallel={\bf E}\cdot \hat{\bf E}^0$,
for a binary medium with dielectric ratio
$\varepsilon_2/\varepsilon_1=5$,
for various fractions $p_2$
of material 2. The space dimension is $d=2$.
With ${\cal{P}}(E)$ given by (\ref{multimod}), and
${\mathbf E}$ written as
${\mathbf E}= E_\parallel\hat{\bf E}^0+{\mathbf E}_\perp$
(which defines ${\mathbf E}_\perp$),
${\cal{P}}_\parallel$ is obtained as
\begin{equation}
{\cal{P}}_\parallel(E_\parallel/E^0)
=E^0\int d^{d-1}\! E_\perp\,
{\cal{P}}({\mathbf E})
=E^0 \sum_\alpha p_\alpha
\frac{1}{(2\pi C^\alpha_\parallel)^{1/2}}
e^{-\frac{1}{2} \frac{(E_\parallel-M^\alpha)^2}{C^\alpha_\parallel}},
\end{equation}
where $E_\perp$ is the transverse component of the field:
${\mathbf{E}}_\perp={\mathbf{E}}-{\mathbf{E}}_\parallel$.
The probability density tends towards a Dirac peak, centered
at $E_\parallel/E^0=1$ as $p_2\to 0$ or $p_2\to 1$, since
the field is then equal to the applied field ${\mathbf E}^0$.
For intermediate volume fractions, the bimodal character of
the probability distribution is evident.

Under the same conditions, Fig.\ 2 displays the probability
density of the scaled modulus of the transverse electric field,
$E_\perp/E^0$. This distribution is defined, for $E_\perp>0$,
by
\begin{equation}
{\cal{P}}_\perp(E_\perp/E^0)={E^0}^{d-1}\int dE_\parallel
d\Omega_{{\mathbf E}_\perp}\, {\cal{P}}({\mathbf E})=
E^0 \sum_\alpha p_\alpha \frac{S_{d-1}}
{(2\pi C^\alpha_\perp)^{(d-1)/2}}\left(\frac{E_\perp}
{E^0}\right)^{d-2}e^{-\frac{1}{2} \frac{E_\perp^2} {C^\alpha_\perp}},
\end{equation}
where $S_d=2\pi^{d/2}/\Gamma(d/2)$ is the surface of a
$d$-dimensional unit sphere. Unlike the previous one,
this distribution is not multimodal since the mean value
of the transverse field is $\langle{\mathbf E}_\perp\rangle_\alpha
={\mathbf 0}$ in each phase, whatever the volume fractions.

Fig.\ 3 displays ${\cal{P}}_\parallel(E_\parallel/E^0)$ for a higher
dielectric ratio $\varepsilon_2/\varepsilon_1=1000$.
The concentrations are $p_2=0.12$, $0.24$, $0.36$. These plots
can be compared to that obtained from hard disks simulations by
Cheng and Torquato (CT) in Fig.\ 3 of Ref.\ \cite{CHEN97}, for
the same dielectric contrast and concentrations $p_2=0.2$, $0.4$,
$0.6$. These concentrations differ from ours by a factor $0.6$.
Apart from these differences, the overall features (shape and
heights) of the probability distributions are well rendered:
a Dirac peak close to $E_\parallel=0$ indicates that the field
is almost null in phase $2$, and a widely spread contribution
from phase 1 for the highest $p_2$ indicates an enhancement
of the fluctuations in this phase as the percolation
threshold (for the Bruggeman theory), or the jamming
threshold (for the CT simulations) is approached. The
differences in the values for $p_2$ can be explained by
the fact that the present theory relies on the Bruggeman
effective-medium formula, of relevance for cell-materials
but not really adequate for hard disks. Moreover, the
overall agreement between our distributions and that of
CT is not so good at moderate dielectric contrast: Bruggeman's
theory completed by the gaussian approximation somewhat
overestimates, because of its percolating nature, the width
of the probability distributions. Better adequation with the
CT simulations would probably be obtained with an
effective-medium theory for dielectric-coated inclusions, 
which would prevent percolation. Since our main objective
however is to discuss an effective-medium therory for WNL
composites, such an improvement will not be considered here.

With these results in hand, analytical investigations of the
theory can be carried out in some limiting cases, as well as
numerical ones in more complicated situations. The next
section examines particular WNL potentials.

\section{Weakly nonlinear power-law potential}
\label{wnpp}
As a first application, a WNL local potential of the form
\begin{equation}
w_x(E)=\frac{1}{2}\varepsilon E^2+\frac{1}{4}\chi E^4+O(E^6),
\end{equation}
is considered, where the $E^4$ term is the first in a series of corrections
to the linear behaviour. The corresponding constitutive law is
\begin{equation}
D_i=[\varepsilon+\chi E^2+O(E^4)] E_i.
\end{equation}
This approximation actually is a weak-field one.

\subsection{Effective-medium formula}
\label{scf}
The effective potential takes the form
\begin{equation}
W(E^0)=\frac{1}{2}\varepsilon^0 {E^0}^2+\frac{1}{4}\chi^0 {E^0}^4
+O({E^0}^4),
\end{equation}
where $\varepsilon^0$ is the Bruggeman result, and \cite{STRO88,BERG89}
\begin{equation}
\chi^0=\langle \chi E^4\rangle/{E^0}^4=\sum_\alpha p_\alpha \chi_\alpha
\langle E^4\rangle_\alpha/{E^0}^4.
\end{equation}
The phase averages $\langle E^4\rangle_\alpha$ are carried out with the
help of Wick's theorem which allows to write down expressions for the
integer moments of a {\em centered} vector gaussian distribution by
mere inspection \cite{LEBE92}: with $A_i=E_i-M^\alpha_i$,
\begin{equation}
\langle A_iA_iA_jA_j\rangle_\alpha
=\langle A_iA_i\rangle_\alpha\langle A_jA_j\rangle_\alpha
+2\langle A_iA_j\rangle_\alpha\langle A_iA_j\rangle_\alpha.
\end{equation}
We deduce
\begin{eqnarray}
\langle E^4\rangle_\alpha&=&
\langle E_i E_i E_j E_j\rangle_\alpha\nonumber\\
&=&C_{ii}^\alpha C_{jj}^\alpha
+2C_{ij}^\alpha C_{ij}^\alpha+2{M^\alpha}^2C_{ii}^\alpha
+4M^\alpha_iM^\alpha_jC_{ij}^\alpha+{M^\alpha}^4\nonumber\\
&=&\left[C_\parallel^\alpha+(d-1)C_\perp^\alpha +{M^\alpha}^2\right]^2+2\left[{C_\parallel^\alpha}^2 +(d-1){C_\perp^\alpha}^2+2{M^\alpha}^2C_\parallel^\alpha\right].
\end{eqnarray}
A few simplifications yield the simple result
\begin{equation}
\label{csc}
\chi^0=\left[(1+2/d) y^2+(2-2/d)z^2-2\right]\langle \mu^4\chi\rangle.
\end{equation}

As a first check, we remark that in a non-disordered situation
where $\varepsilon$ and $\chi$ are constant in the medium, we
have $\varepsilon_0=\varepsilon$ and $\mu=1$, so that $y=z=1$;
whence $\chi_0=\chi$, as was expected.

\subsection{Weak-contrast expansion}
A further check for the self-consistent formula (\ref{csc}) consists
in examining its weak-contrast limit. An exact expression is known
for any nonlinear potential, which is first briefly reminded. In
the weak-contrast expansion \cite{BLUM89,PONT92,PONT97,BART98},
the local potentials $w_x(y)$ are assumed to fluctuate weakly
around their mean value $w^{(0)}(y)=\langle w_x(y)\rangle$.
Introducing a bookkeeping parameter $t$ to be set to 1 in the
final results, the contrast, $w^{(1)}_x(y)$, is defined by
\begin{equation}
w_x(y)=w^{(0)}(y)+w^{(1)}_x(y)t,
\end{equation}
so that $\langle w^{(1)}_x(y)\rangle=0$. An expansion for the
effective potential is sought for in the form:
\begin{equation}
W(E^0)=W^{(0)}+W^{(1)}t+W^{(2)}t^2+\cdots
\end{equation}
Then \cite{PONT97},
\begin{equation}
\label{general}
W(E^0)=\langle w(E^0)\rangle-\frac{1}{2}\frac{n_\parallel}
{\varepsilon_\parallel}\left\langle\left[\partial_i w_x^{(1)} (E^0) {\hat{E}^0}_i\right]^2\right\rangle t^2+ O(t^3),
\end{equation}
where
\begin{eqnarray}
\label{depol}
n_\parallel&=&-\int \frac{d\Omega_{\hat{k}}}{S_d}
\frac{r ({\mathbf\hat{k}}\cdot{\mathbf\hat{E}}^0)^2}
{1+(r-1)({\mathbf\hat{k}}\cdot{\mathbf\hat{E}}^0)^2},\\
\partial_i\partial_j w^{(0)}(E^0)&\equiv&
\varepsilon_\parallel{\hat{E}^0}_i{\hat{E}^0}_j
+\varepsilon_\perp(\delta_{ij}-{\hat{E}^0}_i{\hat{E}^0}_j),\\
r&=&\varepsilon_\parallel/\varepsilon_\perp.
\end{eqnarray}
This result is the exact one \cite{PONT97}. An analogous
expansion for the complementary potential $\tilde{W}(D^0)$
defined from the dual homogeneization problem can be written
down \cite{PONT97,BART98}.

Let us therefore set $\varepsilon=\langle \varepsilon\rangle
+\delta\varepsilon\, t$ and $\chi=\langle \chi\rangle+
\delta\chi\,t$. The Bruggeman permittivity $\varepsilon^0$
is exact to second order in the constrast:
\begin{equation}
\varepsilon^0=\langle \varepsilon\rangle
\left[1-\frac{\langle\delta\varepsilon^2\rangle} {d\langle\varepsilon\rangle^2}t^2+O(t^3)\right].
\end{equation}
Expanding (\ref{csc}) to second order in $t$ entails
\cite{BART98}
\begin{equation}
\label{wnlwc}
\chi^0=\langle\chi\rangle
\left\{1+\left[\frac{2(d+8)}{d(d+2)} \frac{\langle\delta\varepsilon^2\rangle} {\langle\varepsilon\rangle^2}-\frac{4}{d} \frac{\langle\delta\varepsilon\,\delta\chi\rangle} {\langle\varepsilon\rangle\langle\chi\rangle}\right]t^2
+O(t^3)\right\},
\end{equation}
which can directly be obtained from a first-order
expansion of (\ref{general}) in $\chi$.

Comparing now formula (\ref{csc}) to the widely used approximation
\begin{equation}
\label{chi02}
\chi^0\simeq\chi^0_2=\sum_\alpha \chi_\alpha
\langle E^2\rangle_\alpha^2/{E^0}^4=y^2\langle \mu^4\chi\rangle,
\end{equation}
we see that the latter is {\em not} exact to second order
in the contrast, save in $d=1$: its expansion indeed reads
\begin{equation}
\chi^0_2=\langle\chi\rangle
\left\{1+\left[\frac{2(d+2)}{d^2}\frac{\langle\delta\varepsilon^2\rangle} {\langle\varepsilon\rangle^2}-\frac{4}{d} \frac{\langle\delta\varepsilon\,\delta\chi\rangle} {\langle\varepsilon\rangle\langle\chi\rangle}\right]t^2+O(t^3)\right\}.
\end{equation}
Hence, the gaussian decoupling which accounts for the vector
character of the electric field, of importance in nonlinear
problems, is superior to the simple approximation $\langle
E^4\rangle\simeq \langle E^2\rangle^2$.

\subsection{Dilute limit}
For a binary medium with two components of constitutive
parameters $(\varepsilon_1,\chi_1)$ and $(\varepsilon_2,\chi_2)$,
the dilute limit is the limiting situation where the volume
fraction $p$ of (e.g.) component 2 is small.
Setting
\begin{equation}
T=\frac{\varepsilon_2-\varepsilon_1}{\varepsilon_2+(d-1)\varepsilon_1},
\end{equation}
an expansion, to first order in $p$ of (\ref{csc}) yields
\begin{equation}
\label{myres}
\chi^0=\chi_1+\left\{(\chi_2-\chi_1)
(T-1)^4+\chi_1T^2\left[2\frac{d(d+8)}{d+2}-4T+T^2\right]\right\}p+O(p^2).
\end{equation}
Bergman computed the exact effective nonlinear
conductivity of a binary medium in the dilute limit,
for spherical inclusions \cite{BERG89,ZHAN94}. His result takes the form
\begin{eqnarray}
\chi^0_{\text{exact}}&=&
\chi_1+\biggl\{(\chi_2-\chi_1) (T-1)^4\nonumber\\
\label{bergres}
&+&\chi_1T^2\left[2\frac{d(d+8)}{d+2} +4\frac{d(d-4)}{d+2}T+\frac{d(3d^2-10d+16)}{3(d+2)}T^2\right]\biggr\}p+O(p^2).
\end{eqnarray}
Compared to (\ref{bergres}), expression (\ref{myres})
becomes exact when $\chi_1=0$ (nonlinear inclusion in
a linear host). It also becomes exact for $d=1$, and
in the limit $d\to\infty$ where the field is $E=E_0$
in each phase so that $\chi^0=\langle \chi\rangle$.
Moreover, it is exact up to order $T^2$, which is
consistent with its correct limiting weak-contrast
behavior. For $d=2$, it is exact up to order $T^3$.
However, the term of order $T^4$ is wrong for
$2\le d<\infty$.
The reason for this misbehaviour is that the dilute limit requires
an exact computation of $\langle E^4\rangle_\alpha$ (to linear order)
\cite{BERG89}, whereas we approximate it through a gaussian average.

On the other hand, expression (\ref{chi02}) yields
\begin{equation}
\chi^0_2=\chi_1+\left\{(\chi_2-\chi_1) (T-1)^4+
\chi_1T^2\left[2(d+2)-4T+T^2\right]\right\}p+O(p^2),
\end{equation}
once again a result less precise than (\ref{myres}).

At the present time, since it requires an exact solution for
the one-body problem, the test of the dilute limit appears to be
the most challenging one for nonlinear effective-medium theories.

This test is illustrated in Fig.\ 4 which displays $\chi^0$, as
calculated from (\ref{csc}) and (\ref{chi02}), against the
concentration $p$ of medium 2, for moderate dielectric contrast
and $d=2$. The thick line segments at $p=0,1$ represent exact
tangents obtained from (\ref{bergres}). The tangent at $p=1$
follows from substituting $(1-p)$ for $p$ and interchanging the
indices $1$ and $2$ in (\ref{bergres}). Though not exact in the
dilute limit, formula (\ref{csc}) yields tangents quite close to
the exact ones. The marked inaccuracy of formula (\ref{chi02})
near $p=0$ and $p=1$ results in a lower value for $\chi^0$ in
the whole concentration range.

\subsection{Percolative behavior}
Before examining the predictions of Equ.\ (\ref{csc}) near the
percolation transition, I first briefly review the critical
behaviour of WNL composites, and discuss a flaw of the
Bruggeman formula in this context.

An insulator/perfect conductor binary mixture undergoes a
percolation transition for a critical metal fraction
$p=p_c$ \cite{CLER90}. For WNL phases, the critical behaviour
of the field fluctuations is now well understood
\cite{WRIG86}. In the limiting situation where
$\varepsilon_1\ll \varepsilon_2$, and for $p\leq p_c$,
one observes a behaviour
\begin{equation}
\varepsilon^0\propto \varepsilon_1(p-p_c)^{-s},
\qquad\chi^0\propto \chi_1(p-p_c)^{-(2s+\kappa')};
\end{equation}
on the contrary, for $p>p_c$, one has
\begin{equation}
\varepsilon^0\propto \varepsilon_2(p_c-p)^{t},\qquad\chi^0
\propto \chi_2(p_c-p)^{(2t-\kappa)},
\end{equation}
where $s$ and $t$ are the superconductivity and conductivity
exponents, and where $\kappa$ and $\kappa'$ are the noise
exponents which characterize an anomalous nonlinear
susceptibility enhancement near the threshold
\cite{RAMM85,WRIG86}. As a consequence of the exact
inequality $\langle E^4\rangle\ge\langle E^2\rangle^2$,
the exponents $\kappa$ and $\kappa'$ are necessarily
positive.

In the Bruggeman formula, $p_c=1/d$, and $\varepsilon_0
\simeq\varepsilon_1/(1-dp)$ for $p<1/d$, $\varepsilon_0
\simeq\varepsilon_2(1-dp)/(1-d)$ for $p>1/d$
($\varepsilon_1\ll \varepsilon_2$), whence $s=t=1$.
Reporting these expressions into (\ref{csc}) yields
$\kappa=\kappa'=0$, a result shared by all
effective-medium formulae based on (\ref{derivscal}).

These values can be compared to exact bounds obtained
for $\kappa$ and $\kappa'$ on the basis of the
links-nodes-blob (LNB) model for electric percolation,
now commonly accepted as a model for e.g.\ random
resistor networks (RRNs). These are \cite{WRIG86}
\begin{mathletters}
\label{bounds}
\begin{eqnarray}
&&(3d-4)\nu-2t+1\le \kappa\le 2(d-1)\nu-t,\\
&&(4-d)\nu-2s+1\le\kappa'\le 2\nu-s,
\end{eqnarray}
\end{mathletters}
(for $d\le 6$; the values for $d>6$ are that of $d=6$.
Moreover $t$ is not defined for $d=1$), where $\nu$ is
the correlation length exponent: $\xi\propto |p-p_c|^{-\nu}$.
Though the latter information is absent from the Bruggeman
theory, an inequality $\nu>0$ must hold in any situation
relevant to percolating systems. In (\ref{bounds}), the
lower bounds must not be greater than the upper ones,
which leads to lower bounds for the exponents $s$ and
$t$ themselves, namely
\begin{eqnarray}
&&s\ge (2-d)\nu+1,\\
&&t\ge (d-2)\nu+1.
\end{eqnarray}
In the Bruggeman theory, $s=t=1$ so that only $\nu=0$
-- a problematic value by itself -- is allowed, save for
$d=2$. Reporting $\nu=0$ into (\ref{bounds}) leads to
$\kappa=\kappa'=-1$, which is unacceptable. For $d=2$,
one obtains $\kappa=\kappa'=2\nu-1$. Compared to the
values $\kappa=\kappa'=0$ of the above effective-medium
theory, this gives $\nu=1/2$, a non-conflicting value,
though lower than the exact result $4/3$.

Of course, the fact that Bruggeman's formula is a poor
approximation near the percolation threshold is well-known.
However, this discussion enlightens a fundamental
inconsistency in the Bruggeman exponent values which
goes beyond mere numerical inadequacy, as long as one 
wishes to estimate the properties of systems obeying
the LNB scheme. It is to be noted that for $d=2$, where 
no inconsistency appears, the exact equality $s=t$ holds
(a consequence of self-duality) \cite{STRA77} and is obeyed
by the Bruggeman exponents.

The above argument is to be brought together with another one
by Bergman. Discussing the failure of a ``non ambiguous''
non-linear Bruggeman-type model, he concluded that bounds
for fluctuations always refer to some particular type of
microstructure, and that no such bounds exist which apply
to all materials \cite{BERG89}.

Because of these problems, comparisons between
effective-medium theories built on the Bruggeman
formula, and simulations on RRNs are expected to
be more significant in dimension $d=2$ (another
reason for preferring the two-dimensional case
as a test-bench is that the bond percolation
threshold $p_c=1/2$ on a square lattice is exactly
reproduced by Bruggeman's formula). In Fig.\ 5 are
displayed formulas (\ref{csc}) and (\ref{chi02})
against $p$, in a high contrast situation
($\varepsilon_2/\varepsilon_1=10000$). The same
trend as in Fig.\ 4 is observed: formula (\ref{csc})
yields a higher estimate. The values for
$\varepsilon_{1,2}$ and $\chi_{1,2}$ are the
same as that used in Fig.\ 3 of a paper by Levy
and Bergman \cite{LEVY94}, where simulations results
on RRNs are reported, and compared to (\ref{chi02}).
The authors remarked that the height of the peak in
$\chi^0$ at $p_c$ was badly underestimated by (\ref{chi02}),
which gives $\chi^0_{\text{peak}}\simeq 0.5$ only. In
their numerical simulations, for a $30\times 30$ network,
the peak height is found to be
$\chi^0_{\text{peak}}\simeq 0.86$. In the
infinite-size limit, this value is expected
to increase somewhat. In Fig.\ 5, the height
of the peak given by (\ref{csc}) is
$\chi^0_{\text{peak}}\simeq 1.0$, which thus
compares fairly well to simulations.

\subsection{Other power-law nonlinearities}
In this section, potentials of the type (\ref{orig})
where \cite{AHAR87}
\begin{equation}
w^{\text{nl}}_x(E)=\frac{\chi(x)}{\gamma+1}E^{\gamma+1}
\end{equation}
are considered. For simplicity, $\gamma$ is assumed to be
constant in the material. The gaussian averages over the
field in each phase are carried out numerically, with the
help of a bi-variate integration routine. One integration
variable is the field modulus $E$, and the other is the
cosine ${\mathbf E}\cdot{\bf E}^0/(E E^0)$. In fig.\ 
\ref{fig6} are shown plots for
 \begin{equation}
\label{directg}
\chi^0=\langle\chi (E/E^0)^{\gamma+1}\rangle,
\end{equation}
computed from gaussian averages, for a moderate
dielectric contrast and various powers $\gamma>1$
ranging from $\gamma=1.5$ to $\gamma=5$, against
the volume fraction $p$ of medium 2 (solid lines).
An exponential enhancement of $\chi^0$ is observed
with increasing $\gamma$. Its origin lies in the
existence of a non-zero probability for $E>E^0$
in the composite, since ${\bf E}^0=\langle{\bf E}\rangle$
by definition. These fluctuations are amplified by the
nonlinearity, while that for which $E<E^0$ are reduced.
The larger the width of the probability distribution,
the larger this enhancement, so that the peak
culminates in the region $p\simeq p_c=1/2$. Once
again, the estimates of the present theory are much
higher than that predicted by the decoupling assumption
\begin{equation}
\label{decouplg}
\chi^0_2=\sum_\alpha p_\alpha \chi_\alpha \langle(E/E^0)^{2}
\rangle_\alpha^{(\gamma+1)/2},
\end{equation}
(dashed lines), especially for large values of $\gamma$.
This is understandable, since the functions to be averaged can
be written $w_x^{\text{nl}}(E)=\chi(x)f(E^2)$, where
$f(z)=z^{(1+\gamma)/2}/(1+\gamma)$. For $\gamma>1$, these functions
are convex, so that $\langle f(z)\rangle\geq f(\langle z\rangle)$. 
Therefore, the decoupling assumption always underestimates the
fluctuations when applied to such potentials. An overestimation
would instead take place with concave potentials.

\section{Conclusion}
\label{c}
In this article, a theory for the nonlinear susceptibility of
weakly nonlinear composites is proposed. This theory is based
on a multimodal gaussian approximation for the overall
probability distribution of the electric field, to linear
order. The parameters which define this distribution (means
and second moments, in each phase of the disordered medium),
are obtained from Bruggeman's theory, whose limitations are
discussed in this context. The present model provides for the
first time an analytical estimate for the probability distribution
of the electric field, for arbitrarily high dielectric contrast,
in percolating media. The resulting effective nonlinear susceptibility
is exact to second order in the contrast, and close to the exact result
in the dilute limit. Significant quantitative improvement is obtained
on previous nonlinear effective-medium theories, even in the percolating
regime, at least in the two-dimensional case. This study emphasizes the
importance of accounting for the vector character of the electric field 
when averaging the local potentials. Improvements of the present theory
could consist in finding a better approximation than the gaussian one
for the components of the probability distribution of the field. More
realistic distributions could be obtained by, e.g., extending the
perturbative theory of Barth\'{e}l\'{e}my and Orland \cite{BART97} to high
dielectric contrasts via self-consistent effective-medium approximations.

\acknowledgements
Stimulating discussions with M.\ Barth\'{e}l\'{e}my and H.\ Orland are gratefully
acknowledged. I also thank H.\ E.\ Stanley and M.\ Barth\'{e}l\'{e}my for their
kind hospitality at the Center for Polymer Studies (Boston University),
where part of this work was performed.
\appendix

\section{Self-consistent anisotropic linear theory with permanent polarization}
\label{sltwpp}

With local potentials of the form (\ref{ferroel}), the homogeneized
potential reads
\begin{equation}
\label{wlin}
\tilde{W}_{\text{lin}}(E^0)=\frac{1}{2}E^0_i \varepsilon^0_{ij}E^0_j
+P^0_i E^0_j +\frac{1}{2}\langle \Delta P_i \mu_{ij} g_{jk} \Delta P_j\rangle,
\end{equation}
where
\begin{eqnarray}
g_{ij}&=&-\int \frac{d\Omega_{\hat{\bf k}}}{S_d} \frac{ k_i k_j}
{k_l \varepsilon^0_{lm} k_m},\\
\mu_{ij}^\alpha&=&[1-g(\varepsilon^\alpha-\varepsilon^0)]^{-1}_{ij},\\
P^0_i&=&\langle P_j\mu_{ji}\rangle,\\
\Delta P_i^\alpha&=&P_i^\alpha-P^0_i,
\end{eqnarray}
and the Bruggeman condition for $\varepsilon^0_{ij}$ is
$\langle \mu_{ij}\rangle=\delta_{ij}$.

Formula (\ref{wlin}) has been derived by means of a functional formalism
\cite{PELL99}, used with a trial potential of the form \cite{PELL99b}
$W^0(E)=(1/2)E_i\varepsilon^0_{ij} E_j+P^0_i E_i+c^0$. Rather than to
give a demonstration which would complicate the present article, the
validity of (\ref{wlin}) is established by looking at its consequences.
First, one can easily check that, for a binary medium, (\ref{wlin})
exactly reduces to formula (2.16) in the work by Ponte Casta\~{n}eda
and Kailasam \cite{PONT97}. Next, the average field per phase,
$\langle E_i\rangle_\alpha=M^\alpha_i$, is (using the symmetry of the
tensors $g$ and $\mu g$; $\mu$ itself is not necessarily symmetric)
\begin{equation}
\label{eav}
M^\alpha_i=\frac{1}{p_\alpha}\frac{\partial  \tilde{W}_{\text{lin}}(E^0)}{\partial P_i^\alpha}=\mu^\alpha_{ij}
( E^0_j+g_{jk}\Delta P_k^\alpha).
\end{equation}
From the definition of $P^0_i$, one deduces that $\langle M\rangle=E^0$,
in agreement with the boundary conditions. Finally, the macroscopic
consitutive relation derived from the effective potential, namely
\begin{equation}
D^0_i=\frac{\partial \tilde{W}_{\text{lin}}(E^0)}{\partial E^0_i}
=\varepsilon_{ij}^0 E^0_j+P^0_i,
\end{equation}
is consistent with the equation
\begin{equation}
D^0_i=\langle \varepsilon_{ij} E_j+P_i\rangle
=\langle\varepsilon^\alpha_{ij} M_j^\alpha +P_i^\alpha\rangle,
\end{equation}
with $M_j^\alpha$ computed with (\ref{eav}). This directly follows
from the equivalent form of the Bruggeman equation,
$\langle \varepsilon_{ij}\mu_{jk} \rangle=\varepsilon^0_{ij}\langle
\mu_{jk}\rangle=\varepsilon^0_{ij}$, and from the identity
$\mu_{ij}^\alpha=\delta_{ij}+\mu_{ik}^\alpha g_{kl}
(\varepsilon_{li}^\alpha-\varepsilon^0_{li})$.

\begin{thebibliography}{99}
\bibitem{AHAR87} A.\ Aharony,
Phys.\ Rev.\ Lett.\ {\bf 58}, 2726 (1987).

\bibitem{STRO88} D.\ Stroud and P.\ M.\ Hui,
Phys.\ Rev.\ B {\bf 37}, 8719 (1988).

\bibitem{ZENG88} X.\ C.\ Zeng and D.\ J.\ Bergman,
P.\ M.\ Hui and D.\ Stroud, Phys.\ Rev.\ B {\bf 38}, 10970 (1988).

\bibitem{BERG89} D.\ J.\ Bergman, Phys.\ Rev.\ B {\bf 39}, 4598 (1989).

\bibitem{BLUM89} R.\ Blumenfeld and D.\ J.\ Bergman,
Phys.\ Rev.\ B {\bf 40}, 1987 (1989); Phys.\ Rev.\ B {\bf 44}, 7378 (1991);
D.\ J.\ Bergman and R.\ Blumenfeld, Phys.\ Rev.\ B {\bf 54}, 9555 (1996).

\bibitem{KOTH90} N.\ C.\ Kothari, Phys.\ Rev.\ A {\bf 41}, 4486 (1990).

\bibitem{PONT92} P.\ Ponte Casta\~{n}eda, G.\ deBotton and G.\ Li,
Phys.\ Rev.\ B {\bf 46}, 4387 (1992).

\bibitem{GU92} G.\ Q.\ Gu and K.\ W.\ Yu,
Phys.\ Rev.\  B {\bf 46}, 4502 (1992).

\bibitem{YU93} K.\ W.\ Yu, P.\ M.\ Hui and D.\ Stroud,
Phys.\ Rev.\ B {\bf 47}, 14150 (1993); Phys.\ Rev.\ B {\bf 56}, 14195 (1997).

\bibitem{YU94} K.\ W.\ Yu, Y.\ C.\ Chu and E.\ M.\ Y.\ Chan,
Phys.\ Rev.\ B {\bf 50}, 7984 (1994).

\bibitem{LEE95} H.\-C.\ Lee, K.\ W.\ Yu and G.\ Q.\ Gu,
J.\ Phys.: Condens.\ Matter {\bf 7}, 8785 (1995).

\bibitem{GAO96} L.\ Gao, Z.-Y.\ Li,
Phys.\ Lett.\ A {\bf 219}, 324 (1996).

\bibitem{WAN96} W.\ M.\ V.\ Wan, H.\ C.\ Lee, P.\ M.\ Hui, and K.\ W.\ Yu,
Phys.\ Rev.\ B {\bf 54}, 3946 (1996).

\bibitem{SALI97} L.\ Sali and D.\ J.\ Bergman,
J.\ Stat.\ Phys.\ {\bf 86}, 455 (1997).

\bibitem{PONT97} P.\ Ponte Casta\~{n}eda and M.\ Kailasam,
Proc.\ R.\ Soc.\ London A, 793 (1997).

\bibitem{BART98} M.\ Barth\'{e}l\'{e}my and H.\ Orland,
Eur.\ Phys.\ J.\ B {\bf 6}, 537 (1998), cond-mat/9806302.
In this reference, a slight error has led to the erroneous
weak-contrast expansion (37) for $\chi^0$: the minus sign
should be a $+$ in Eqs.\ (33a), (33c), and the sign of the
second order term in (35) is a $+$. One then retrieves our
result (\ref{wnlwc}).

\bibitem{DUXB86} P.\ M.\ Duxbury, P.\ D.\ Beale and P.\ M.\ Leath,
Phys.\ Rev.\ Lett.\ {\bf 57}, 1052 (1986).

\bibitem{BERG78} D.\ J.\ Bergman, Phys.\ Rep.\ {\bf 43}, 377 (1978).

\bibitem{HUI95a} P.\ M.\ Hui, Y.\ F.\ Woo and W.\ M.\ V.\ Van,
J.\ Phys.: Condens.\ Matter {\bf 7}, L593 (1995).

\bibitem{HILL63} R.\ Hill,
J.\ Mech.\ Phys.\ Solids {\bf 11}, 357 (1963).

\bibitem{PONT92b} P.\ Ponte Casta\~{n}eda,
SIAM J.\ Appl.\ Math.\ {\bf 52}, 1321 (1992), and references therein.

\bibitem{BART97} M.\ Barth\'{e}l\'{e}my and H.\ Orland,
Phys.\ Rev.\ E {\bf 56}, 2835 (1997), and references therein.

\bibitem{DUXB95} P.\ M.\ Duxbury, M.\ Beale and C.\ Moukarzel,
Phys.\ Rev.\ B {\bf 51}, 3476 (1995).

\bibitem{CHEN91} Z.\ Chen and P.\ Sheng,
Phys.\ Rev.\ B {\bf 43}, 5735 (1991).

\bibitem{CHEN97} H.\ Cheng and S.\ Torquato,
Phys.\ Rev.\ B {\bf 56}, 8060 (1997).

\bibitem{CULE98} D.\ Cule and S.\ Torquato, cond-mat/9805030 (1998).

\bibitem{WILL81} J.\ Willis, Adv.\ Appl.\ Mech.\ {\bf 21}, 1 (1981).

\bibitem{NOTE1} It can be shown that the boundary condition
$\langle{E}\rangle=E^0$ truly corresponds to the more
standard condition of uniform normal induction on the
boundary $\partial V$, namely $n_i D_i=n_i D^0_i$
(see, e.g.\ Ponte-Casta\~{n}eda \cite{PONT92b}); we then
have $\langle{D}\rangle=D^0$; but the relationship
(\ref{constithom}) between $D_0$ and $E_0$ is known
only after the homogeneization problem has been solved.

\bibitem{PONT98} P.\ Ponte Casta\~{n}eda and P.\ Suquet,
Adv.\ Appl.\ Mech.\ {\bf 34}, 171 (1998).

\bibitem{BRUG35} D.\ A.\ G.\ von Bruggeman,
Ann.\ Phys.\ (Leipzig) {\bf 24}, 636 (1935);
R.~Landauer, {\it J.~Appl.~Phys.} {\bf 23}, 779 (1952).
See also R.~Landauer, in {\it Electical Transport and 
Optical Properties of 
Inhomogeneous Media}, edited by J.~C.~Garland and D.~B.~Tanner.
AIP Conf.~Proc.~No.\ 40 (AIP, New York, 1978).

\bibitem{LEBE92} M.\ Le Bellac and G.\ Barton (translator)
{\em Quantum and Statistical Field Theory}
(Oxford University Press, Oxford, 1992).

\bibitem{CLER90} J.\ P.\ Clerc, G.\ Giraud, J.\ M.\ Laugier
and J.\ M.\ Luck, Adv.\ Phys.\ {\bf 39}, 191 (1990),
and references therein.

\bibitem{ZHAN94} X.\ Zhang and D.\ Stroud,
Phys.\ Rev.\ B {\bf 49}, 944 (1994).

\bibitem{RAMM85} R.\ Rammal, J.\ Phys.\ (Paris) Lett.\ {\bf 46}, L129 (1985);
R.\ Rammal, C.\ Tannous, P.\ Breton and A.-M.\ S.\ Tremblay,
Phys.\ Rev.\ Lett.\ {\bf 54}, 1718 (1985).

\bibitem{WRIG86} D.\ C.\ Wright, D.\ J.\ Bergman and Y.\ Kantor,
Phys.\ Rev.\ B {\bf 33}, 396 (1986).

\bibitem{STRA77} J.\ P.\ Straley,
Phys.\ rev.\ B {\bf 15}, 5733 (1977).

\bibitem{LEVY94} O.\ Levy and D.\ J.\ Bergman,
Phys.\ Rev.\ B {\bf 50}, 3652 (1994).

\bibitem{PELL99} Y.-P.\ Pellegrini and M.\ Barth\'{e}l\'{e}my, cond-mat/0001223, to be published.

\bibitem{PELL99b} Y.-P.\ Pellegrini, unpublished.

\end{thebibliography}

\twocolumn
\begin{figure}
\narrowtext
\vspace*{0.0cm}
\centerline{ \epsfysize=0.9\columnwidth{\rotate[r]{\epsfbox{fig1_c.eps}}}}
\vspace*{0.0cm}
\caption{Field probability density function  ${\cal{P}}_\parallel(E_\parallel/E^0)$  ($E_\parallel={\bf E}\cdot\hat{\bf E}^0$) in the  Bruggeman approximation, for various volume fractions  $p_2$ of component 2. The dielectric constant ratio is  $\varepsilon_2/\varepsilon_1=5$. Dots: $p_2=0.05$  (violet,highest curve); $p_2=0.25$ (blue,lowest curve).  Solid line: $p_2=0.5$ (green). Dashes: $p_2=0.75$  (orange, lowest curve); $p_2=0.95$ (red, highest curve). }
\label{fig1}
\end{figure}

\begin{figure}
\narrowtext
\vspace*{0.0cm}
\centerline{ \epsfysize=0.9\columnwidth{\rotate[r]{\epsfbox{fig2_c.eps}}}}
\vspace*{0.0cm}
\caption{Field probability density function  ${\cal{P}}_\perp(E_\perp/E^0)$  ($E_\perp=|\!|{\bf E}-E_\parallel{\bf E}^0|\!|$)  in the Bruggeman approximation, for various volume  fractions $p_2$ of component 2. The dielectric  constant ratio is $\varepsilon_2/\varepsilon_1=5$.  Dots: $p_2=0.05$ (violet, highest curve); $p_2=0.25$  (blue, lowest curve). Solid line: $p_2=0.5$ (green). Dashes:  $p_2=0.75$ (orange, lowest curve); $p_2=0.95$ (red, highest curve). }
\label{fig2}
\end{figure}

\begin{figure}
\narrowtext
\vspace*{0.0cm}
\centerline{ \epsfysize=0.9\columnwidth{\rotate[r]{\epsfbox{fig3_c.eps}}}}
\vspace*{0.0cm}
\caption{Field probability density function  ${\cal{P}}_\parallel(E_\parallel/E^0)$  ($E_\parallel={\bf E}\cdot\hat{\bf E}^0$) in  the Bruggeman approximation, for various volume  fractions $p_2$ of component 2. The dielectric  constant ratio is $\varepsilon_2/\varepsilon_1=1000$.  Solid: $p_2=0.12$; dashes: $p_2=0.24$; dots: $p_2=0.36$.\\ \\ \\ }
\label{fig3}
\end{figure}

\begin{figure}
\narrowtext
\vspace*{0.0cm}
\centerline{ \epsfysize=0.9\columnwidth{\rotate[r]{\epsfbox{fig4.eps}}}}
\vspace*{0.3cm}
\caption{Comparison between Eqs.\ (\ref{csc})  (solid line) and (\ref{chi02}) (dashes). Effective  nonlinear susceptibility vs.\ volume fraction $p$ of  component 2. The space dimension is $d=2$. The  parameters are: $\varepsilon_1=1$, $\varepsilon_2=10$,  $\chi_1=0.01$, $\chi_2=0.1$. The segments at $p=0$ and  $p=1$ represent exact slopes computed with expression  (\ref{bergres}).}
\label{fig4}
\end{figure}

\begin{figure}
\narrowtext
\vspace*{0.0cm}
\centerline{ \epsfysize=0.9\columnwidth{\rotate[r]{\epsfbox{fig5.eps}}}}
\vspace*{0.0cm}
\caption{Comparison between Eqs.\ (\ref{csc})  (solid line) and (\ref{chi02}) (dashes). Effective  nonlinear susceptibility vs.\ volume fraction $p$ of  component 2. The space dimension is $d=2$. The parameters  are: $\varepsilon_1=1$, $\varepsilon_2=10^4$, $\chi_1=10^{-4}$,  $\chi_2=0.1$. The segment at $p=1$ represents the exact slope  computed with expression (\ref{bergres}). }
\label{fig5}
\end{figure}

\begin{figure}
\narrowtext
\vspace*{0.0cm}
\centerline{ \epsfysize=0.9\columnwidth{\rotate[r]{\epsfbox{fig6_c.eps}}}}
\vspace*{0.3cm}
\caption{Comparison between Eqs.\ (\ref{directg})  (solid line) and (\ref{decouplg}) (dashes) for power-law  nonlinearities $w^{\text{nl}}_x=\chi(x) E^{\gamma+1}/(\gamma+1)$.  Effective nonlinear susceptibility vs.\ volume fraction $p$ of  component 2. The space dimension is $d=2$. The parameters are:  $\varepsilon_1=1$, $\varepsilon_2=10$, $\chi_1=0.01$,  $\chi_2=0.1$. From bottom to top: $\gamma=1.5$ (blue), $2$ (cyan),  $3$ (turquoise), $4$ (green), $5$ (black). }
\label{fig6}
\end{figure}

\end{document}