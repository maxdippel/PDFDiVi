% this is the original unexpurgated version

\magnification1200\baselineskip14pt minus .5pt


\leftline{To the editors of Physics Today:}
\bigskip

N.D.Mermin's reference frame (Phys Today, May 1991, page 91) advocated a
formal electronic bulletin board system as a means of circumventing recognized
inadequacies of journals. Most physicists undoubtedly regarded the merit of  
his proposal as so self-evident that it required no comment. The printed
replies (Phys Today, Jan 1992, page 13) were primarily concerned with the
sociological issues of individuals desperately clinging to a soon-to-be
outmoded status quo and the implicit plight of institutions whose terrain
is about to be defined out from under them. The technological issues, on
the other hand, are already settled and the demise of certain journals,
welcome or otherwise, has been underway for more than a decade.

THEORY:\hfill\break
The lead letter by Schultz for example exposes a bizarre degree of
computer anxiety and ignorance that may well be representative of some
unfortunate subset of otherwise educated physicists (including the Physics
Today editors who chose to feature such sentiments so prominently in their
letters column). It is first of all not clear why issues concerning
refereeing and revisions should be classified as hardware problems, but
the irrelevance of refereed journals to ongoing research in the elementary
particle theory community has long been recognized. At least since the mid
1970's, the preprint distribution system has been the primary means of
communication of new research ideas and results. We have learned to
determine on the basis of title and abstract (and occasionally authors)
whether we wish to read a paper, and to verify any result that we need
rather than rely on the alleged verification of overworked or otherwise
careless referees (such as ourselves on bad days). The small amount of
filtering provided by refereed journals plays no effective role in our
research. As far as revisions, a system that allows ongoing corrections or
addenda is manifestly preferable to one that does not. It is not difficult
to implement a system in which people who so desire can be assured that
they are kept up to date (and those who do not are not bothered).

The first true hardware issue, that of mass storage, no longer poses a real
obstacle. Given that an average paper (without figures) requires 50 kbytes
(roughly 100 blocks) to store, one of the current generation of rapid
access gigabyte disk drives costing under \$2000 could hold 20,000 papers at
an average cost of 10 cents / paper. Slower access media for archival
storage cost even less: a miniature video 8 cartridge available from
discount electronic dealers for under \$7 can hold 2 gigabytes, i.e.\ 40,000
such papers. The data equivalents of many years worth of most journals are
a small fraction of what is routinely handled by many experimentalists on a
daily basis, and the costs of data storage are certain to continue to
diminish for the foreseeable future.

Since the storage is so inexpensive, it could be duplicated at multiple
distribution points, minimizing the risk of loss due to accident or
catastrophe, and facilitating worldwide network access. Backup procedures
to preserve data do not ordinarily restrict access to modern computer
systems, and ``electrical storms in Iowa'' do not affect distributed
network access any more than they affect our ability to telephone to
another continent. The current NSFnet runs 24 hours a day with virtually
no interruptions in service, and transfers data at a rate of 1.5 Mbyte/sec
(1/30 sec per paper). (Even the effective transfer rate of 100 kbyte/sec
between Los Alamos and Paris during peak operating periods is adequate for
most purposes.) The currently projected upgrades (Physics Today, Jan 1992,
page 54) to 45 Mbyte/sec in five years and a few Gbyte/sec within a decade
will then be more than adequate to deal with increased usage.

Software for treating figures has not yet been standardized, but the vast
majority of networked Physics institutions have screen previewers and
laserprinters that display and print postscript files created by a wide
variety of graphics programs. In the near future, high resolution digital
scanners will be as commonplace as Fax machines have rapidly become, so
that figures produced by any means will be accommodated. With appropriate
data compression and/or postscript conversion, papers with figures would
typically increase the storage requirements cited above by an
inconsequential factor of two. Conformity to a uniform computer
standard in order to communicate results to the largest possible audience
should pose no greater burden than is already imposed on the majority of
the world that communicates using a non-native English language.

In the long term, electronic access to scientific research will be a major
boon to developing countries since the expense of connecting to an
existing network is infinitesimal compared to that of constructing,
stocking, and maintaining libraries. The trend experienced over the past
decade in the western world, in which data transmission lines have become as
common as telephone service and terminals and laserprinters as common
as typewriters and photocopy machines, could be repeated even more quickly
as countries in eastern europe and the third world develop an electronic
infrastructure. In the short term they will be no worse off than they
already are, receiving information via conventional means from the nearest
redistribution point.

Concerns about interference from malevolent hackers are hopelessly exaggerated.
The difference between executable and text files is known to managers of
most systems, if not to their users. An archive is easily rendered immune
to corruption, and minimal precautions can assure users of bulletin board
systems that their own system resources are not endangered. Anonymous FTP
servers running on internet have for years allowed the academic community
open exchange of executable software that is far more susceptible to
malfeasance, but nonetheless have adopted simple measures that insure
their own safety and that of their users.

It is equally straightforward to implement charges for such a system if
one so desires, either via a flat access rate or a rate depending on
monitored usage. (After all, the phone companies manage somehow, and
accounting systems for mainframe computers used to be commonplace.)
Piggybacking on existing network resources, however, such a system would
cost so little to set up and maintain it could be offered virtually for
free. Overburdened terminal resources at libraries are not an issue, since
the access would typically be via the terminal or workstation on one's desk.

EXPERIMENT:\hfill\break
So rather than write a note to Physics Today complaining about its
impossibility, I chose to spend a couple of afternoons last summer writing
the software to assess the feasibility of fully automating such a system.
The software is rudimentary, and allows users to communicate email
requests to the internet address hepth@xxx.lanl.gov (hepth = high energy
physics theory). Remote users may submit and replace papers, get papers,
get listings, get help on available commands, search the listings for
authornames, etc. The system also allows anonymous FTP access to the
papers and listings directories. No papers have been lost or corrupted.
The papers are submitted as TeX source files, already the unique mode of
producing papers within the target community. Figures are typically
submitted as postscript files or produced within the TeX or LaTeX picture
environments.

For the initial user base, it was possible to take advantage of
pre-existing informal email distribution lists in the subject of 2d
gravity and conformal field theory (that had arisen in spontaneous
response to the inadequacy of conventional distribution systems). Starting
from a subscriber list of 160 addresses in mid-August 1991, the system had
grown after five months to serve over 750 subscribers covering every
continent (except Antarctica), and the numbers still have not levelled
off. At the same time the subject range has broadened to encompass most of
formal quantum field theory and string theory.

The system sends out a daily listing of titles and abstracts of new papers
received (via a listserv facility volunteered by the Slac library, where
papers are also registered into the Spires database). It receives an
average of 2.5 new papers / day, responds to an average of 300 requests / day, 
and transmits more than 600 copies of papers on peak days. Internet
access time to correctly configured mailers or FTP servers is typically a
few seconds. Numerous institutional addresses have subscribed to receive
every paper automatically so that their secretaries can acquire valuable
practice texing papers to place on preprint shelves.

The system runs as a background job on a small Unix workstation that is
primarily used for other purposes, and poses no noticeable drain on cpu
resources. It will require about 50Mb / year (i.e.\ $<$ \$100/year) for
storing papers, including figures. Its network usage is $< 10^{-5}$ of the
lanl.gov backbone, so as well poses a negligible drain on local network
resources. It requires very little intervention and recently ran
functionally unattended for five weeks while I was abroad. It could easily
be scaled up by one or two orders of magnitude in usage on the basis of
existing (i.e.\ borrowed) resources. It is difficult to estimate the
potential for a dedicated system of the future only because we are so far
from saturating the resources of the current experimental one (which is run
free of charge).

There is little reason to believe that the users of this system are
particularly computer literate, but its simple-minded use of conventional
email protocol has made it instantly accessible. The system in its present
form was not intended to replace journals, but only to organize a
haphazard and unequal distribution of electronic preprints. It is
increasingly being used as an electronic journal, however, given that it
is frequently far more rapid to retrieve electronically a file of a paper
than to retrieve physically a paper from a file cabinet. Besides
minimizing geographic inequalities by eliminating the boat-mail gap
between continents, it also institutes a form of democracy in research in
which anyone from beginning graduate student on up in principle has equal
access to new results. No longer is it crucial to have the correct
connections or to be on exclusive mailing lists in order to be kept
informed of progress in one's field. The pernicious problem of lost or
stolen preprints experienced by some large institutions is as well
definitively exorcised. Communication among colleagues at the same
institution may also be enhanced, since it is relatively common that they
cross-request one another's preprints from the remote server (the reasons
for which in general I hesitate to contemplate).

Systems such as the one I describe are already being implemented in other
disciplines. Moreover the current software will be made freely available
to anyone who has the disk and network resources to support a bulletin
board. These systems are still primitive, and only tentative first steps
in the ultimate correct direction. The obvious next step is a local
menu-driven interface which would automatically connect to the nearest
central server and transparently pipe selected papers through text
formatters directly to a screen previewer or printer. (Such software is
already running on my local network, and parallels widely available
software that provides network access to centralized services, e.g.\
weather.) Perhaps the centralized databases and further software
development may ultimately be taken over and systematized by publishing
institutions such as the AIP, the British IOP (which has expressed
interest), and others, if they are prescient enough to reconfigure
themselves for the inevitable.
\smallskip
\hfill\vbox{\hsize 3in Bulletin Board \quad  1/92\par
Los Alamos National Laboratory\par
hepth@xxx.lanl.gov\par}
\smallskip
\noindent Note (to be included): In the same January issue of Physics Today, 
the editors express undue pride in having actually received a letter by email,
a surprise only since an email address is nowhere provided (except
evidently to the editors of Physical Review D from which it was obtained
for the purpose of sending the current message). Researchers in
elementary particle and string theory routinely include electronic
addresses in every publication (a trend begun by a paper of mine written
with two colleagues more than four years ago).

\bye
