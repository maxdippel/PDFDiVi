%%% ======================================================================
%%%  @LaTeX-file{
%%%     filename        = "template.tex",
%%%     version         = "3.0",
%%%     date            = "October 20, 1992",
%%%     ISO-date        = "1992.10.20",
%%%     time            = "15:41:54.18 EST",
%%%     author          = "Optical Society of America",
%%%     contact         = "Frank E. Harris",
%%%     address         = "Optical Society of America
%%%                        2010 Massachusetts Ave., N.W.
%%%                        Washington, D.C.  20036-1023",
%%%     email           = "fharris@pinet.aip.org (Internet)",
%%%     telephone       = "(202) 416-1903",
%%%     FAX             = "(202) 416-6120",
%%%     supported       = "yes",
%%%     archived        = "pinet.aip.org/pub/revtex,
%%%                        Niord.SHSU.edu:[FILESERV.REVTEX]",
%%%     keywords        = "REVTeX, version 3.0, template, Optical
%%%                        Society of America",
%%%     codetable       = "ISO/ASCII",
%%%     checksum        = "62445 133 697 5395",
%%%     docstring       = "This is the Optical Society of America
%%%                        template for creating a manuscript under
%%%                        REVTeX 3.0 (release of November 10, 1992).
%%%
%%%                        The checksum field above contains a CRC-16
%%%                        checksum as the first value, followed by the
%%%                        equivalent of the standard UNIX wc (word
%%%                        count) utility output of lines, words, and
%%%                        characters.  This is produced by Robert
%%%                        Solovay's checksum utility."
%%% }
%%% ======================================================================
%%%%%%%%%%%%%%%%%% file template.tex %%%%%%%%%%%%%%%%%%%%
%                                                       %
%    Copyright (c) Optical Society of America, 1992.    %
%                                                       %
%%%%%%%%%%%%%%%%%%% November 17, 1992 %%%%%%%%%%%%%%%%%%%
%
% THIS FILE IS A TEMPLATE TO PRODUCE AN ARTICLE SUBMISSION
% TO THE OSA JOURNALS, JOSA-A, JOSA-B, and APPLIED OPTICS.
%
% THIS TEMPLATE CONTAINS TYPESETTING COMMANDS WHICH BEGIN WITH A
% BACKSLASH.  THESE COMMANDS WILL BE READ BY LATEX, USING THE
% REVTEX 3.0 STANDARD MACROS.   PLEASE FILL IN THE REQUIRED DATA
% FOR THE MACROS, BUT DO NOT ALTER THE DEFINITIONS.
%
% EXAMPLE: IN \author{Authors' names} , PLEASE FILL IN THE
% AUTHORS' NAME(S).
%
% COMMENTS BEGIN WITH THE PERCENT (%) SYMBOL. AFTER A %, ANY
% DATA ON THE REST OF A LINE WILL NOT PRINT.
%
\documentstyle[osa,manuscript]{revtex}  % DON'T CHANGE
%
%
\newcommand{\MF}{{\large{\manual META}\-{\manual FONT}}}
\newcommand{\manual}{rm}        % Substitute rm (Roman) font.
\newcommand\bs{\char '134 }     % add backslash char to \tt font
%
%
\begin{document}                % INITIALIZE - DONT CHANGE
%
%
%
\title{Insert your title here}
\author{Insert your names here}
\address{Insert the name of your University, company or Institute here}
% \author{}   % Use this and the next line only if there is a second
% \address{Another University, etc.}  % address. (Remove the left % marks)
%
\maketitle
\begin{abstract}                % DON'T CHANGE THIS LINE
 Insert your abstract here.  No {\it REVTEX} limit to number of lines.
\end{abstract}
%
\section{Introduction}               % Introduction goes below.
Introduction goes here...
 Most TeX, {\it REVTEX} and LaTeX commands begin with a backslash.
 See the manual for a list of special, non-printing characters
 (Appendix B and Table 5).

\section{Insert your section title here.}
 Copy the above line to head each section.  Section 2 text
 goes below.
 Please use the \verb+\verb+ command to print passages with
 special characters in the text.  Surround the special
 characters with "+" signs.
 Please use the \verb+\cite{RefTag}+\cite{RefTag} command to
 cite references in the text.

 Create equations, lists, and tables using standard {\it REVTEX} 3.0
 macros.  See the manual and examples for further instructions.
 Please place tables and figure captions at the end of your
 manuscript.  Examples are shown below.  I feel these examples
 are necessary because the spelling and syntax of some
 commands have changed, compared to earlier versions of {\it REVTEX}.
%
% ({\it REVTEX} 3.0 automatically issues
% a \newpage command when the \begin{table} or \begin{figure}
% commands are used, so the figures and tables will be placed
% on separate pages by {\it REVTEX}).




 \begin{references}
% Please use the \bibitem command to create references.
 \bibitem{RefTag}Author, "Title," Journal {\bf Volume,}
 page numbers (year).  %(Format for Journal Reference)
 \end{references}

 \begin{figure}  % Please send figures with disk, or separately if
% if it is an e-mail submission. (Good photo or India ink drawing.)
 \caption{Please place your figure caption here.}
 \end{figure}

 \begin{table}
 \caption{Please place your table caption here.}
 \begin{tabular}{lrcd} % In second brace, l = left, r = right,
% c = centered and d = decimal justification.
 One&Two&Three&Four\\  % Separate items with &. End line with \\
 \tableline % Creates a horizontal line.
 One&Two\tablenote{footnote.}&Three&Four\\ % Place \tablenote{}
% after item to be footnoted.
 \end{tabular}
 \end{table}


\end{document}

% end of file Template.tex
