%%		REFORDER.TEX			6/7/85	Doug E.
%%
%%	This macro package is intended for use with JNL.
%%	It will automatically order and sort the references in a paper
%%	by order of first citation.(!!)  To use, say \input reforder
%%	after \input jnl (and after all definitions of \refto etc.,
%%	in particular after any use of the \refstyleXX macros),
%%	but before any use of \refto etc.  Use \refto{} (or \ref{} and
%%	\Ref{}) to cite references in the text.  Use \refis{} to list
%%	the references,  SKIP A LINE after each reference.  Open the
%%	reference listing with \references and close it with \endreferences.
%%
%%	REFORDER depends on the
%%	JNL macros \refto{}, \ref{}, \Ref{} to identify citation of references.
%%	REFORDER also contains a macro \cite{} which can be used to cite
%%	references; e.g., ``Reference \cite{19} blah...'' will produce
%% 	output ``Reference 19 blah''.  Multiple citations can be separated
%%	by commas.  E.g., \refto{24,26,27} and \cite{3,7}
%%	are legal.  Also legal is \refto{3-7}, which expands to mean the same
%%	as \refto{3,4,5,6,7}.  Reference ``numbers'' can in general be any
%%	alphanumeric string; e.g. BjorkenAndDrell is perfectly OK used in
%%	the form\ref{BjorkenAndDrell};  such strings should contain no blanks.
%%
%%	If you have your own pet macros to cite references such as, e.g.,
%%	\def\referpet#1{$^(#1)$)}, you can bring it to the attention of
%%	REFORDER so all \referpet's will be properly cited simply by
%%	declaring ``\citeall\referpet'' once near the beginning, after
%%	\referpet is defined and
%%	after \input reforder.  This has the effect of redefining the macro
%%	as e.g., \def\referpet#1{$^(\cite{#1})$}.  (Such \citeall'ed macros
%%	must have exactly one argument #1, as in \referpet.)  See e.g.,
%%	the end of this file where \refto, \ref and \Ref are \citeall'ed.
%%
%%	REFORDER depends on the macro \refis{} for listings.  In fact,
%%	a reference listing can appear anywhere in the paper after
%%	its first citation.  The macro \endreferences actually triggers
%%	sorting and listing of references.  Skip a line after a reference
%%	listing (or, alternatively, end each listing in \par).

\catcode`@=11
\newcount\r@fcount \r@fcount=0
\newcount\r@fcurr
\immediate\newwrite\reffile
\newif\ifr@ffile\r@ffilefalse
\def\w@rnwrite#1{\ifr@ffile\immediate\write\reffile{#1}\fi\message{#1}}

\def\writer@f#1>>{}
\def\referencefile{%			  Stuff to write .REF file
  \r@ffiletrue\immediate\openout\reffile=\jobname.ref%
  \def\writer@f##1>>{\ifr@ffile\immediate\write\reffile%
    {\noexpand\refis{##1} = \csname r@fnum##1\endcsname = %
     \expandafter\expandafter\expandafter\strip@t\expandafter%
     \meaning\csname r@ftext\csname r@fnum##1\endcsname\endcsname}\fi}%
  \def\strip@t##1>>{}}
\let\referencelist=\referencefile

\def\citeall#1{\xdef#1##1{#1{\noexpand\cite{##1}}}}
\def\cite#1{\each@rg\citer@nge{#1}} % Variable No. of args, separated by ","

\def\each@rg#1#2{{\let\thecsname=#1\expandafter\first@rg#2,\end,}}
\def\first@rg#1,{\thecsname{#1}\apply@rg}	% each@ag is a general purpose
\def\apply@rg#1,{\ifx\end#1\let\next=\relax%	  variable no. of arg. macro.
\else,\thecsname{#1}\let\next=\apply@rg\fi\next}% args separated by commas

\def\citer@nge#1{\citedor@nge#1-\end-}	% Check for M-N range (M and N numbers)
\def\citer@ngeat#1\end-{#1}
\def\citedor@nge#1-#2-{\ifx\end#2\r@featspace#1 % Single argument
  \else\citel@@p{#1}{#2}\citer@ngeat\fi}	% M-N range of arguments
\def\citel@@p#1#2{\ifnum#1>#2{\errmessage{Reference range #1-#2\space is bad.}
    \errhelp{If you cite a series of references by the notation M-N, then M and
    N must be integers, and N must be greater than or equal to M.}}\else%
 {\count0=#1\count1=#2\advance\count1 by1\relax%
\expandafter\r@fcite\the\count0,%
  \loop\advance\count0 by1\relax%	  Loop from M to N
    \ifnum\count0<\count1,\expandafter\r@fcite\the\count0,%
  \repeat}\fi}

\def\r@featspace#1#2 {\r@fcite#1#2,}	% Eat spaces at beginning or end of arg
\def\r@fcite#1,{\ifuncit@d{#1}		% Cite individual reference
    \expandafter\gdef\csname r@ftext\number\r@fcount\endcsname%
    {\message{Reference #1 to be supplied.}\writer@f#1>>#1 to be supplied.\par
     }\fi%
  \csname r@fnum#1\endcsname}

\def\ifuncit@d#1{\expandafter\ifx\csname r@fnum#1\endcsname\relax%
\global\advance\r@fcount by1%
\expandafter\xdef\csname r@fnum#1\endcsname{\number\r@fcount}}

\let\r@fis=\refis			% Save old \refis, redefine
\def\refis#1#2#3\par{\ifuncit@d{#1}%      Use two params #2 #3 to strip blank
    \w@rnwrite{Reference #1=\number\r@fcount\space is not cited up to now.}\fi%
  \expandafter\gdef\csname r@ftext\csname r@fnum#1\endcsname\endcsname%
  {\writer@f#1>>#2#3\par}}

\def\r@ferr{\endreferences\errmessage{I was expecting to see
\noexpand\endreferences before now;  I have inserted it here.}}
\let\r@ferences=\references
\def\references{\r@ferences\def\endmode{\r@ferr\par\endgroup}}

\let\endr@ferences=\endreferences
\def\endreferences{\r@fcurr=0%		  Save old \endreferences, redefine
  {\loop\ifnum\r@fcurr<\r@fcount%	  Loop over refnum and produce text
    \advance\r@fcurr by 1\relax\expandafter\r@fis\expandafter{\number\r@fcurr}%
    \csname r@ftext\number\r@fcurr\endcsname%
  \repeat}\gdef\r@ferr{}\endr@ferences}

% Save old \endpaper, redefine it to write parting message.

\let\r@fend=\endpaper\gdef\endpaper{\ifr@ffile
\immediate\write16{Cross References written on []\jobname.REF.}\fi\r@fend}

\catcode`@=12

\citeall\refto		% These macros will generate citations
\citeall\ref		%
\citeall\Ref		%
