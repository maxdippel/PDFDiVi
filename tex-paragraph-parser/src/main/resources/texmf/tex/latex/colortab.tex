%% BEGIN colortab.tex/colortab.sty
%%
\def\fileversion{0.9}
\def\filedate{93/01/11}
%%
%% COPYRIGHT 1993, by Timothy Van Zandt, tvz@Princeton.EDU
%%
%% DESCRIPTION:
%%   colortab.tex/colortab.sty lets you shade or color the cells in a
%%   alignment environments (e.g., \halign and \LaTeX's tabular and
%%   array environments). It is compatible with most TeX macro packages,
%%   including Plain TeX, LaTeX, AmSTeX, and Ams-LaTeX. You must be
%%   use in conjunction with other color commands.
%%
%% INSTALLATION:
%%   Put this file where your TeX looks for inputs, under the name
%%   colortab.tex. Name a copy colortab.sty to use as a LaTeX style option,
%%   or create a file colortab.sty with the lines:
%%     \input colortab.tex
%%     \endinput
%%
%% DOCUMENTATION:
%%   See colortab.doc, which might be appended to this file.
%%
%% COPYING:
%%   Copying of part or all of this file is allowed under the following
%%   conditions only:
%%   (1) You may freely distribute unchanged copies of the file. Please
%%       include the documentation when you do so.
%%   (2) You may modify a renamed copy of the file, but only for personal
%%       use or use within an organization.
%%   (3) You may copy fragments from the file, for personal use or for
%%       distribution, as long as credit is given where credit is due.
%%
%%   You are NOT ALLOWED to take money for the distribution or use of
%%   this file or modified versions or fragments thereof, except for
%%   a nominal charge for copying etc.
%%
%% CODE:
%
% colortab.tex uses TeX primitives, plus the Plain TeX commands:
%   \newcount, \newtoks, \newbox, \newif and \z@.
%
% The LaTeX specific macros also use some LaTeX commands.
%
\csname ColortabLoaded\endcsname
\let\ColortabLoaded\endinput
%
\message{\space\space v\fileversion\space\space\filedate\space\space <tvz>}
\edef\TheAtCode{\the\catcode`\@}\catcode`\@=11
%
% These scratch registers are set globally:
%
\newcount\CT@cnt
\newdimen\CT@dim
\newbox\CT@box
%
% This flag is used to shut off colortab:
%
\newif\ifcolortab
\colortabtrue
%
% To color row cells of unknown height we first have to put the rows in an
% alignment and measure the resulting height, then put down a row of blobs
% (with \hrule, and then move back up the page.
%
% For TeX's \halign primitive, \coloralign saves the preamble in the token
% register \CT@preamble. This is used to calculate the height of the columns
% that are to be colored in.
%
\newtoks\CT@preamble
\def\coloralign{%
  \afterassignment\coloralign@
  \setbox\CT@box=\hbox}
\def\coloralign@#1\cr{%
  \egroup
  \CT@preamble{\halign\bgroup#1\cr}%
  \the\CT@preamble}
%
% \CC reads the next row, which should contain the color commands, and then
% the rows to be colored. The height of the latter rows is calculated, and
% then the row of colors is processed. \CT@columncolor uses \leaders to get
% each color to fill the column horizontally, and it uses the previously
% calculated height to fill the columns vertically.
%    \begin{macrocode}
\def\CT@columncolor{%
  \ifx\next\@nil
    \def\next##1&\@nil{\gdef\CT@next{##1}}%
    \expandafter\next\CT@next\@nil
    \def\CT@temp\@nil{}%
  \else
    \ifx\next\CT@space
      \def\CT@temp{%
        \def\next{\futurelet\next\CT@columncolor}%
        \afterassignment\next\let\CT@temp= }%
    \else
      \def\CT@temp{\CT@@columncolor}%
    \fi
  \fi
  \CT@temp}
\def\CT@@columncolor#1&{%
  \ifx\next&%
    \expandafter\gdef\expandafter\CT@next\expandafter{\CT@next\omit{}&}%
  \else
    \expandafter\gdef\expandafter\CT@next\expandafter{\CT@next
      \omit{{\ignorespaces#1{\leaders\hrule height\CT@dim\hfil}}}&}%
  \fi
  \futurelet\next\CT@columncolor}
{\def\\{\global\let\CT@space= }\\ }
\def\CC@#1#2#3{%
  \ifcolortab
    \noalign{%
      \setbox\CT@box=\vbox{#1#3\crcr\egroup}%
      \CT@dim=\ht\CT@box
      \global\advance\CT@dim by \dp\CT@box
      \def\CT@next{}%
      \futurelet\next\CT@columncolor#2&\@nil}%
    \CT@next\cr
    \noalign{\vskip-\CT@dim}%
  \fi
  #3}
\def\CC#1\cr#2\ECC{\CC@{\the\CT@preamble}{#1}{#2}}
%
% With \LaTeX's alignment macros, the preamble is saved for us in \@preamble.
% Otherwise, \LCC is the same as \CC.
%
\def\LCC#1\\#2\ECC{\CC@{\@preamble}{#1}{#2}}
%
% This unaligned color stuff only works with PSTricks:
%
\def\AC#1{%
  \@ifundefined{color@#1}%
    {\@pstrickserr{Color not defined: `#1'}\@ehpa}%
    {\pstVerb{\pst@usecolor{#1}}}}
\def\EAC{\pstVerb{\pst@currentcolor}}
\def\NAC#1{\noalign{\AC{#1}}}
\def\ENAC{\noalign{\EAC}}
%
\def\SP{\noalign{\global\let\CT@savedpreamble\@preamble}}
\def\RP{\noalign{\global\let\@preamble\CT@savedpreamble}}
%
\newcount\CT@LCnum
\def\LColors#1\\{%
  \noalign{%
    \ifnum\CT@LCinitflag=\c@LT@tables\else\CT@LCinit\fi
    \ifnum\CT@LCflag=\z@
      \global\CT@LCnum 1
      \CT@LColors#1&\@nil
      \global\CT@LCnum\CT@LCmax
    \fi}}
\def\CT@LColors#1&{%
  \global\@namedef{LColor@\the\CT@LCnum}{#1}%
  \@ifnextchar\@nil{\@gobble}{%
    \global\advance\CT@LCnum 1
    \CT@LColors}}
\def\CT@LCinit{%
  \ifcolortab
    \@ifundefined{LT@\romannumeral\c@LT@tables}%
      {\gdef\CT@LCflag{1}}{\gdef\CT@LCflag{\z@}}%
  \else
    \gdef{CT@LCflag{1}}%
  \fi
  \ifnum\CT@LCflag=\z@
    \begingroup
      \def\omit\kern##1&{%
        \advance\CT@LCnum 1
        \CT@dim=##1\relax
        \expandafter\xdef\csname LCdim@\the\CT@LCnum\endcsname{%
          \number\CT@dim sp}}%
      \CT@LCnum\z@
      \@nameuse{LT@\romannumeral\c@LT@tables}&%
      \xdef\CT@LCmax{\the\CT@LCnum}%
    \endgroup
  \fi
  \xdef\CT@LCinitflag{\the\c@LT@tables}}
\def\CT@LCinitflag{\z@}
\def\LC@Cell{%
  \kern-\@tempdima
  \ifnum\CT@LCflag=\z@
    \ifnum\CT@LCnum=\CT@LCmax\relax
      \global\CT@LCnum 1
    \else
      \global\advance\CT@LCnum 1
    \fi
    \CT@dim=\@nameuse{LCdim@\the\CT@LCnum}\relax
    \advance\CT@dim-\@tempdimb
    \expandafter\ifx\csname LColor@\the\CT@LCnum\endcsname\@empty\else
      \begingroup
        \@nameuse{LColor@\the\CT@LCnum}{%
        \vrule width\CT@dim}%
      \endgroup
      \kern-\CT@dim
    \fi
  \fi}
\def\LC{\@tempdima=\z@\@tempdimb=\z@\LC@Cell}
\def\LCi{\LCi@\LC@Cell}
\def\LCi@{%
  \@tempdima=\doublerulesep
  \ifnum\CT@LCflag=\z@
    \ifnum\CT@LCnum=\CT@LCmax\relax
      \@tempdimb=\arrayrulewidth
    \else
      \@tempdimb=\z@
    \fi
  \fi}
\def\LCii{%
  \@tempdima=\z@
  \@tempdimb=\doublerulesep
  \advance\@tempdimb=\arrayrulewidth
  \LC@Cell}
\def\LCiii{%
  \LCi@
  \advance\@tempdimb\doublerulesep
  \advance\@tempdimb\arrayrulewidth
  \LC@Cell}
\def\LCz#1#2{%
  \@tempdima=#1
  \@tempdimb=#2
  \LC@Cell}
%
\catcode`\@=\TheAtCode\relax
\endinput
%% END colortab.tex/colortab.sty
