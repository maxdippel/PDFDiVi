%%% ======================================================================
%%%  @LaTeX-file{
%%%     filename        = "manaip.tex",
%%%     version         = "3.0",
%%%     date            = "November 10, 1992",
%%%     ISO-date        = "1992.11.10",
%%%     time            = "15:41:54.18 EST",
%%%     author          = "American Institute of Physics",
%%%     contact         = "Peggy Sutherland
%%%                        and
%%%                        Liz Belmont",
%%%     address         = "500 Sunnyside Blvd.
%%%                        Woodbury, NY 11797",
%%%     telephone       = "(516) 576-2454
%%%                        and
%%%                        (516) 576-2335",
%%%     email           = "liz@aip.org (Internet)
%%%                        and
%%%                        peggys@aip.org",
%%%     supported       = "yes",
%%%     archived        = "pinet.aip.org/pub/revtex,
%%%                        Niord.SHSU.edu:[FILESERV.REVTEX]",
%%%     keywords        = "REVTeX, version 3.0, input guide,
%%%                        American Institute of Physics",
%%%     codetable       = "ISO/ASCII",
%%%     checksum        = "03170 459 1827 15427",
%%%     docstring       = "This is the American Institute of Physics chapter
%%%                        in the input guide for REVTeX 3.0.
%%%
%%%                        The checksum field above contains a CRC-16
%%%                        checksum as the first value, followed by the
%%%                        equivalent of the standard UNIX wc (word
%%%                        count) utility output of lines, words, and
%%%                        characters.  This is produced by Robert
%%%                        Solovay's checksum utility."
%%% }
%%% ======================================================================
\documentstyle[aps]{revtex}
\makeatletter
%run page numbers by chapter
\def\thepage{3-\@arabic\c@page}
%these page numbers need abit more width
\def\@pnumwidth{2em}
\def\REVTeX{REV\TeX}
\makeatother
\begin{document}
\twocolumn
\title{REV\TeX\ Information For AIP Authors\\
\vskip 1pc
Instructions to authors for preparing compuscripts to be submitted to\\
AIP journals in REV\TeX\ 3.0 format}

\maketitle
\tableofcontents

\makeatletter
\global\@specialpagefalse
\def\@oddhead{\REVTeX{} 3.0\hfill Released November 10, 1992}
\let\@evenhead\@oddhead
% run page numbers by "chapter", with copyright for first page
\def\@oddfoot{\reset@font\rm\hfill \thepage\hfill
\ifnum\c@page=1
  \llap{\protect\copyright{} 1992
  American Institute of Physics}%
\fi
} \let\@evenfoot\@oddfoot
\makeatother

\section{Introduction}

The American Institute of Physics will be translating REV\TeX\ files into
code that can be used by our Xyvision composition system.  Composition
(page make-up) will take place using the Xyvision-coded file. For this
reason, compuscripts will look exactly like conventionally processed
articles when they appear in print, and authors should check proofs
carefully.

After the REV\TeX\ file is converted to Xyvision code, the article will
follow conventional processing schedules.  After sufficient experience with
compuscript processing, the AIP will examine whether page-charge reductions
and/or accelerated production schedules for compuscripts are possible.


\section{Participating Journals }

The American Institute of Physics publishes more than twenty journals at
the time this Guide was drafted.  The following journals are published by
the American Institute of Physics and most are covered in this Guide.


\begin{itemize}
      \item Astronomical Journal$^*$
      \item American Journal of Physics
      \item Applied Physics Letters$^*$
      \item Chaos
      \item Computers in Physics
      \item Journal of Applied Physics
      \item The Journal of the Acoustical Society of America
      \item The Journal of Chemical Physics
      \item Journal of Mathematical Physics
      \item Journal of Rheology
      \item Journal of Vacuum Science and Technology A
      \item Journal of Vacuum Science and Technology B
      \item Medical Physics
      \item Publications of the Astronomical Society of the Pacific
      \item Physics of Fluids A
      \item Physics of Fluids B
      \item Powder Diffraction
      \item Review of Scientific Instruments
\end{itemize}

$^*$The {\em Astronomical Journal} and {\em Applied Physics Letters} are
not part of this compuscript program. See Sec. \ref{basic} for more
information on the {\em Astronomical Journal}. The {\em Applied Physics
Letters} may be added to the compuscript program in the future; its weekly
nature will make it more challenging.


\section{Where to turn for help}

In this section are listed both AIP contacts and editorial office contacts.
Authors can always contact AIP with any questions, technical or procedural,
surrounding the compuscript program.  However, answers may be obtained more
quickly by contacting the editorial office in question, depending on what
stage the manuscript is in and whether the question is procedural or
technical (involving the use of REV\TeX ).


\subsection{Prior to manuscript acceptance}
\subsubsection{Procedural questions}

Each editorial office has internal procedures to meet its needs.  At the
time this guide went to press, procedures for the editorial offices were
being documented in the interest of publishing specific instructions for
the authors at a future time.

For now, procedures may change frequently; therefore, every effort should
be made to address procedural questions that arise prior to acceptance
(such as, will you be contacted when the  REV\TeX\ file is needed, can the
file be sent via electronic mail, etc.) with the editorial office in
question rather than with AIP production staff.

However,  if author questions cannot be properly resolved by the editorial
office, contact AIP as directed below.

{\em Electronic Submissions.}  Some editorial offices may be equipped to
receive files for review purposes.  Please contact the editorial office in
question to determine the office's capability and procedures.





\subsubsection{REV\TeX\ Questions}

Questions about REV\TeX\ use are best directed to AIP staff. See Sec.
\ref{contacts} below.

\subsection{After Manuscript Acceptance}

Contact AIP staff with any questions that arise after the manuscript has
been accepted for publication.
\subsection{AIP Contacts\label{contacts}}



There are two contacts at AIP for compuscript processing questions: Peggy
Sutherland and Liz Belmont (both are at 500 Sunnyside Boulevard, Woodbury,
NY 11797).  Please put questions in writing whenever possible or use
electronic mail.

\begin{verse}
Liz Belmont\\
516/576-2454\\
e-mail: liz@aip.org\\
\end{verse}
\vspace{1pc}

\begin{verse}
Peggy Sutherland\\
516/576-2335\\
e-mail: peggys@aip.org\\
\end{verse}

\subsection{Editorial Office Contacts}

Editorial office contacts are listed below. In the pages that follow, the
journal editor is listed first, and followed in parentheses by the office
staff member (if available) who is able to answer procedural questions.

\smallskip

\noindent      Astronomical Journal$^*$

\begin{verse}
             Dr. Paul Hodge (Chaim Rosemarin)\\
             Dept. of Astronomy, FM-20\\
             Univ. of Washington\\
             206/685-2150\\
             ASTRO@phast.phys.washington.edu\\
             $^*$Do not use REV\TeX .  See Sec. \ref{basic}.
\end{verse}

\noindent      American Journal of Physics

\begin{verse}
             Dr. Robert Romer (Karla Keyes)\\
             Merrill Science Bldg., Rm  222\\
             P.O. Box 2262\\
             Amherst College\\
             Amherst, MA  01002\\
             413/542-5792\\
\end{verse}

\noindent      Chaos

\begin{verse}
             Dr. David Campbell\\
             (Janis Bennett, AIP; 516/576-2403)\\
             Univ. of Illinois, Urbana-Champaign\\
             Loomis Lab. of Physics\\
             1110 W. Green Street\\
             Urbana, IL  61801\\
             217/333-3760\\
             dkc@faust.physics.uiuc.edu\\
\end{verse}

\noindent      Computers in Physics

\begin{verse}
             Dr. Lewis Holmes\\
             (Param Bajwa: 202/745-1895)\\
             AIP\\
             1630 Connecticut Ave. NW\\
             Washington, DC 20009\\
\end{verse}

\noindent      Journal of Applied Physics

\begin{verse}
             Dr. Steven Rothman (Cathey Dial)\\
             Argonne Natl. Lab.\\
             P.O. Box 8296\\
             Argonne, IL  60439-8296\\
             708/252-4200\\
\end{verse}

\noindent      The Journal of the Acoustical Society of America

\begin{verse}
             Dr. Daniel Martin\\
             $[$Peggy Sutherland, AIP; 516/576-2335$]$\\
             7349 Clough Pike\\
             Cincinnati, OH  45244\\

\end{verse}
\newpage
\noindent      The Journal of Chemical Physics

\begin{verse}
             Dr. John C. Light (Mitty Collier)\\
             James Franck Inst.\\
             Univ. of Chicago\\
             5735 S. Ellis Avenue\\
             Chicago, IL  60637\\
             312/702-7067\\
\end{verse}


\noindent      Journal of Mathematical Physics

\begin{verse}
             Dr. Roger G. Newton\\
             (Hank Davis, 812/855-3576)\\
             Indiana University\\
             The Poplars, Rm 324\\
             Bloomington, IN  47405\\
             812/855-2095\\
\end{verse}

\noindent      Journal of Rheology

\begin{verse}
             Dr. Arthur Metzner (Debbie Cook, x2299)\\
             Center for Composite Materials\\
             Univ. of Delaware\\
             Newark, DE  19716\\
             302/831-2328\\
\end{verse}


\noindent      Journal of Vacuum Science and Technology A

\begin{verse}
             Dr. Gerald Lucovsky (Becky York)\\
             Dept. of Physics\\
             North Carolina State Univ.\\
             Raleigh, NC  27650\\
             919/248-1861\\
\end{verse}

\noindent      Journal of Vacuum Science and Technology B

\begin{verse}
             Dr. Gary McGuire (Becky York)\\
             Microelectronics Center of North Carolina\\
             P.O. Box 12889\\
             Research Triangle Park, NC  27709\\
             919/248-1861\\
\end{verse}

\noindent      Medical Physics

\begin{verse}
             Dr. John S. Laughlin\\
             (Linda Addonisio, 212/639-7414)\\
             Memorial Sloan-Kettering Cancer Center\\
             Medical Physics\\
             P.O. Box 62\\
             1275 York Ave.\\
             New York, NY  10021\\
\end{verse}

\noindent      Publications of the Astronomical Society of the Pacific

\begin{verse}
             Dr. Howard Bond (Denise Dankert)\\
             Space Telescope Science Institute\\
             3700 San Martin Dr.\\
             Baltimore, MD  21218\\
             410/338-4958\\
\end{verse}

\noindent      Physics of Fluids A

\begin{verse}
             Dr. Andreas Acrivos (Tammy Erickson)\\
             The Levich Inst., Steinman 202\\
             City College of New York\\
             Convent Ave. at 140th St.\\
             New York, NY  10031\\
             212/283-0962\\
\end{verse}

\noindent      Physics of Fluids B

\begin{verse}
             Dr. Ronald C. Davidson (Ellen Webster)\\
             Plasma Physics Lab.\\
             James Forrestal Campus\\
             Princeton Univ.\\
             P.O. Box 451\\
             Princeton, NJ  08543\\
             609/243-2425\\
\end{verse}

\noindent      Powder Diffraction

\begin{verse}
             Deane K. Smith\\
             Dept. of Geosciences\\
             Pennsylvania State Univ.\\
             239 Deike Bldg.\\
             University Park, PA  16802\\
             815/865-5782
\end{verse}

\noindent      Review of Scientific Instruments

\begin{verse}
             Dr. Thomas H. Braid\\
             Argonne National Laboratory\\
             P.O. Box 8293\\
             Argonne, IL  60439-8293\\
             708/252-8236\\
\end{verse}


\section{Other packages that can be used with REV\TeX}

BIBTEX can be used to produce the bibliography section.  Use the guidelines
for {\em Physical Review}.

{\em PostScript}.  If PostScript figures are available, provide them in
files separate from the REV\TeX\ file. Label the files FIG1, FIG2, FIG3a,
etc.  PostScript files may or may not be used for the production process.
Glossy, camera- ready figures must be provided to the editorial office for
production even if PostScript figures are available.


\section{Basic Information}\label{basic}

For most journals, the guidelines issued for APS and OSA compuscripts are
acceptable.  The exceptions are as follows:
\begin{itemize}

\item{\bf For the {\em Astronomical Journal}:} Do not use REV\TeX .  The
AAS is developing a macro package for use at this journal's editorial
office.  AIP will announce a compuscript program for AJ articles based on
the AAS macros when software development has been completed.

\item If the journal requires the author-year format for reference
citations in text, as in the {\em Journal of Rheology}, do not use {\tt
cite} in the text.  Type the reference explicitly within the text, and type
the references in alphabetical order within the reference section. For
example, in text:


      ... is the relaxation spectrum of the Rouse chain model
      [Bloomfield and Zimm (1966)]...



\item {\bf Macro advice:} Type-1 macros (those created by the author
exclusively to save keystrokes) are discouraged. Although some macros may
be acceptable in ``reasonable'' numbers and when simply constructed,
authors may find that their compuscript has been rejected from the
author-prepared program if macros cause complications in processing.


\item {\bf More macro advice:} Any time author-created macros are included
in the file, the compuscript will require special handling, therefore
reducing the time saved in production.  Use of macros by authors will
necessarily affect the feasibility of reducing page charges and of
accelerated production schedules for compuscripts. Also, we would recommend
strongly that author proofs are carefully proofread to ensure that macros
have been properly processed.

\item {\bf More macro advice:} If we cannot process a file due to problem
macros, the manuscript will be prepared conventionally to avoid production
delays.

\item {\bf Even more macro advice:} We will not process author macros that
give instructions to the \TeX\ program (those that change internal
counters, for example) or that involve recursive math definitions.  If in
doubt, remove macros before providing the file.


\item {\bf Aligning math:} Do not attempt to break and align math.  The
Xyvision composition system will automatically break and align displayed
equations as needed.
      \end{itemize}


\section{How To Use REV\TeX}

AIP's translation software will handle any compuscript that is formatted by
following the REV\TeX\ guidelines provided by the American Physical Society
or the Optical Society of America.  Review the manuscript samples provided
by APS and OSA.  Select a manuscript format based on your requirements; be
sure to review Basic Information (Sec. \ref{basic}) above to ensure that
your format choice is applicable.

In future releases of this guide, we may be able to recommend guidelines
based on the AIP journal where the submission is being made.  At this time,
however, be assured that any REV\TeX\ style selected will be supported by
the AIP.

\end{document}
